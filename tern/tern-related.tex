\section{Related Work} \label{sec:tern-related}

\noindent
{\bf \smt and \dmt systems.} Although \tern provides determinism, it 
differs from existing \dmt systems~\cite{dmp:asplos09, coredet:asplos10, 
kendo:asplos09} by making threads stable, \ie, repeating familiar behaviors 
across different inputs. Another difference is that \tern reduces timing 
nondeterminism for server programs through the windowing approach.

The closest system to \tern in this category is
Kendo~\cite{kendo:asplos09}, a software-only \dmt system that also
enforces synchronization orders instead of memory access orders for
efficiency.  \coredet~\cite{coredet:asplos10} is another software-only \dmt
system that enforces deterministic memory access orders.  Both systems are
based on logical clocks and have been shown to work on scientific
benchmarks, such as \splash.  The authors of \coredet have noted that a small
modification to the original program leads to a much different
\coredet-instrumented program, which the idea of schedule memoization may
address.  \coredet is a software implementation (with extensions) of
DMP~\cite{dmp:asplos09}, a hardware \dmt system .

Grace~\cite{grace:oopsla09} proposes a novel approach to making C and C++
programs with fork-join parallelism behave like sequential programs.  It
runs each thread within a process and commits memory writes atomically and
deterministically.  It detects memory access conflicts efficiently using
hardware page protection.  Grace has been shown to perform and scale well
on \phoenix benchmarks~\cite{phoenix-benchmarks} and a Cilk~\cite{cilk}
benchmark.  Unlike Grace, \tern aims to make general multithreaded
programs, not just fork-join programs, deterministic and stable.

\noindent
{\bf Deterministic Replay} Deterministic
replay~\cite{r2:osdi,friday2007,srinivasan:flashback,revirt,dejavu,
vmware-record-replay,smp-revirt:vee08,pres:sosp09,scribe:sigmetrics10,odr:sosp09
,capo:asplos09}
aims to replay the exact recorded executions, whereas \tern ``replays"
memoized schedules on different inputs.  Some recent deterministic replay
systems include Scribe, which tracks page ownership to enforce
deterministic memory access~\cite{scribe:sigmetrics10}; Capo, which defines
a novel software-hardware interface and a set of abstractions for
efficient replay~\cite{capo:asplos09}; PRES and ODR, which
systematically search for a complete execution based on a partial
one~\cite{pres:sosp09,odr:sosp09}; and SMP-ReVirt, which uses clever page
protection trick for recording the order of conflicting memory
accesses~\cite{smp-revirt:vee08}.

\noindent
{\bf Concurrency Errors} The complexity in developing multithreaded
programs has led to many concurrency errors~\cite{lu:concurrency-bugs}.  A
significant number of them are not data races, but atomicity and order
errors~\cite{lu:concurrency-bugs}, which can be deterministically
reproduced or avoided using only synchronization orders.

Much work exists on concurrency error
detection~\cite{yu:racetrack:sosp,savage:eraser,racerx:sosp03,lu:muvi:sosp,
avio:asplos06,conmem:asplos10},
diagnosis~\cite{racefuzzer:pldi08,ctrigger:asplos09,atomfuzzer:fse08}, and
correction~\cite{dimmunix:osdi08,gadara:osdi08}.  \tern aims to make the
executions of multithreaded programs deterministic and stable, and is
complementary to existing work on concurrency errors.  Specifically,
\tern can use existing work to detect and fix the errors in the schedules
it selects.  Moreover, even for programs free of concurrency errors,
\tern still provides value by making their behaviors repeatable. 

\noindent {\bf Symbolic Execution} The combination of symbolic and
concrete executions has been a hot research topic.  Researchers have built
scalable and effective symbolic execution systems to detect
errors~\cite{dart:pldi,cute:fse,godefroid:grammar-fuzzing,
godefroid:whitebox-fuzzing,klee:osdi08,yang:malicious-disk:oakland06,
cadar:exe:ccs06,s2e:hotdep09,taas:socc10},
block malicious inputs~\cite{castro:bouncer}, and preserve privacy in
error reports~\cite{castro:bug-report-privacy}.  Compared to these
systems, \tern applies symbolic execution to a new domain: tracking input
constraints to reuse schedules.
