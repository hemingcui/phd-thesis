\section{Background}
\label{sec:tern-background}

This section first explains why the instability
problem is inherent in existing \dmt systems (\S\ref{sec:dmt-background}),
and then our choice of schedule representation in \tern
(\S\ref{sec:tern-define-schedule}).

\subsection{The Instability Problem in \dmt} \label{sec:dmt-background}

A \dmt system is, conceptually, a function that maps an input $I$ to a
schedule $S$.  The properties of this function are that the same $I$
should map to the same $S$ and that $S$ is a feasible schedule for
processing $I$.  A stable \dmt system such as \tern has an additional
property: it maps similar inputs to the same schedule.  Existing \dmt
systems, however, tend to map similar inputs to different schedules, thus
suffering from the instability problem.  

We argue that this problem is inherent in existing \dmt systems because
they are stateless.  They must provide the same schedule for an input
across different runs, using information only from the current run.
To force threads to communicate (\eg, acquire locks or access shared memory)
deterministically, existing \dmt systems cannot rely on physical clocks. 
Instead, they
maintain a logical clock per thread that ticks deterministically based on
the code this thread has run.  Moreover, threads may communicate only when
their logical clocks have deterministic values (\eg, smallest across the
logical clocks of all threads~\cite{kendo:asplos09}).  By induction,
logical clocks make threads deterministic.

However, the problem with logical clocks is that for efficiency,
they must tick at
roughly the same rate to prevent a thread with a slower clock from
starving others.  Thus, existing \dmt systems have to tie their logical
clocks to low-level instructions executed (\eg, completed
loads~\cite{kendo:asplos09}).  Consequently, a small change to the input or
execution environment may alter a few instructions executed, in turn
altering the logical clocks and subsequent thread communications.  That is,
a small change to the input or execution environment may cascade into a
much different (\eg, correct vs. buggy) schedule.  


\subsection{Schedule Representation}
\label{sec:tern-define-schedule}

Typical \smt or \dmt systems have considered two types of schedules: (1) a
deterministic order of shared memory
accesses~\cite{dmp:asplos09,coredet:asplos10} and (2) a synchronization
order (\ie, a total order of synchronization operations)~\cite{kendo:asplos09}. 
The first type of schedules are fully deterministic even if there are races, but
they are costly to enforce on commodity hardware (\eg, up to 10 times
overhead~\cite{coredet:asplos10}).  The second type can be efficiently enforced
(\eg, 16\% overhead~\cite{kendo:asplos09}) because most code is not
synchronization code and can run in parallel; however, they are deterministic
only for inputs that lead to race-free runs~\cite{recplay:tocs,kendo:asplos09}.

\tern represents schedules as synchronization orders for efficiency.  An
additional benefit is that synchronization orders can be reused more
frequently than memory access orders (cf next subsection).
Moreover, researchers have found that many concurrency errors
are not data races, but atomicity and order
violations~\cite{lu:concurrency-bugs}.  These errors can be
deterministically reproduced or avoided using only  synchronization orders.

Although data races may still make runs which reuse schedules nondeterministic,
\tern is less prone to this problem than existing \dmt
systems~\cite{kendo:asplos09} because it has the flexibility to select
schedules.  If it detects a race in a memoized schedule, it can simply
discard this schedule and memoize another.  This selection task is often
easy because most schedules are race-free.  In rare cases, \tern may be
unable to find a race-free schedule, resulting in nondeterministic runs.
However, we argue that input nondeterminism cannot be fully eliminated
anyway, so we may as well tolerate some scheduling nondeterminism,
following the end-to-end argument.



