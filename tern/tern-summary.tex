\section{Summary}
\label{sec:tern-summary}

We have presented \tern, the first \smt and \dmt system that makes general
multithreaded programs stable by repeating the same schedules on different
inputs.  \tern does so using schedule memoization: if a schedule is shown
to work on an input, \tern memoizes the schedule; if a similar input
arrives later, \tern simply reuses the memoized schedule. \tern is also the
first \dmt system to mitigate input timing nondeterminism for server
programs.

Our \tern implementation runs on Linux.  It requires no new hardware, no
modifications to the underlying OS or synchronization library, and only a
few lines of modifications to the multithreaded programs.  We evaluated
\tern on a diverse set of real programs, including two server programs, one
desktop program, and 11 scientific programs.  Our results show that
\tern is easy to use, makes programs more deterministic and stable, and has
reasonable overhead.  \tern is the first \smt and \dmt system shown to work on
applications as large, complex, and nondeterministic as \mysql and \apache.
It demonstrates that \smt and \dmt have the potential to greatly improve
understanding, testing, and debugging of multithreaded programs, making these
programs much easier to get right.

