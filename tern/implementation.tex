\section{Refinements} \label{sec:tern-impl}

This section describes four refinements we made, one for
determinism (\S\ref{sec:tern-detect-race}) and three for speed
(\S\ref{sec:tern-skip-waits}-\S\ref{sec:tern-slicing}).

\subsection{Detecting Data Races} \label{sec:tern-detect-race}

%%\capstartfalse
\begin{figure}
\begin{minipage}[t]{0.45\linewidth}
\tiny
\lgrindfile{tern/code/avoided-race.cpp}
\caption{\small\em A conventional race, not a schedule race.}
\label{fig:tern-avoided-race}
\end{minipage}
\hfill
\begin{minipage}[t]{0.48\linewidth}
\tiny 
\lgrindfile{tern/code/symbolic-race.cpp}
\vspace{-.2in}
\caption{\small\em A symbolic race that occurs only when $i=j$.}
\label{fig:tern-symbolic-race}
\end{minipage}
\vspace{-.2in}
\end{figure}
%%\capstarttrue

As discussed in \S\ref{sec:tern-define-schedule}, if a memoized schedule allows
data races, runs reusing this schedule may become nondeterministic.  Thus,
for determinism, we would like to detect races in memoized schedules and
discard them from the schedule cache.  A general race detector would flag
too many races for \tern because it detects conventional races with respect
to the original synchronization constraints of the program, whereas we
want to detect races with respect to the order constraints of a
schedule~\cite{recplay:tocs} (call them \emph{schedule races}).
Figure~\ref{fig:tern-avoided-race} shows a conventional race, but not a
schedule race because the synchronization order shown ``kills'' the race.

Thus, we built a simple race detector to detect schedule races.  It runs
with the memoizer and is happens-before based.  It considers one memory
access happens before another with respect to the synchronization order
the memoizer records.  Sometimes a pair of instructions may appear to be a
race, when in fact their relative order does not alter a run.  For
instance, a write-write race is benign if both instructions write the same
value.  Similarly, a read-write race is benign if the value written by one
instruction does not affect the value read by another.  Our race detector
prunes these benign races.

Our detector also flags \emph{symbolic races}, the races that are
data-dependent on inputs.  Figure~\ref{fig:tern-symbolic-race} shows an example.
Both variables $i$ and $j$ are inputs, and the race occurs only when $i =
j$.  The risk of a symbolic races is that it may be absent in a
memoization run and thus skip detection, but show up nondeterministically
in a reuse run.  To detect symbolic races, our race detector queries the
underlying symbolic execution engine for pointer equality.  For example,
to detect the race in Figure~\ref{fig:tern-symbolic-race}, it would query the
underlying symbolic execution engine for the satisfiability of
$\&a[i]=\&a[j]$.  It flags a symbolic race if this constraint is satisfiable.
Once a symbolic race is flagged, \tern adds additional input constraints to
ensure that the race does not occur in reuse runs.  For
Figure~\ref{fig:tern-symbolic-race}, we would add $\&a[i]\neq \&a[j]$, which
simplifies to $i\neq j$.

Our race detector can detect all schedule races in a memoization run.  It
can also detect all symbolic races if developers correctly annotate all
data that affect synchronization operations and memory locations accessed.
If this assumption holds and our race detector reports no races in a
memoization run, \tern ensures that the memoized schedule can be
deterministically reused.

%% If a symbolic race is
%% possible, we add additional input constraints to prevent the race from
%% occurring in a reuse run.  For instance, for the race in
%% Figure~\ref{fig:symbolic-race}, we will add $i \neq j$ to the input
%% constraints of the memoized schedule.  
% Our race detector is sound if the \v{symbolic()} annotations capture all
% nondeterminism that may alter the memory locations accessed.


%% As discussed in \S\ref{sec:define-schedule}, reusing a schedule may lead
%% to nondeterministic runs if the schedule allows races.  Thus, for
%% determinism, we would like to detect races when memoizing schedules and
%% discard racy schedules from the schedule cache.  A general race detector
%% would flag too many races for \tern because it detects conventional races
%% with respect to the original synchronization constraints of the program,
%% whereas we want to detect races with respect to the order constraints of a
%% schedule (call them \emph{schedule races}).  Figure~\ref{fig:avoided-race}
%% shows a conventional race, but not a schedule race because the
%% synchronization order shown prevents the race from occurring.  Thus, we
%% built a simple custom race detector that runs within the memoizer.  It
%% detects schedule races with respect to the synchronization order logged by
%% the memoizer.

%% To prune benign races that do not make schedules nondeterministic, we
%% implemented another technique to check that reordering two racy
%% instructions would not alter the run.  Specifically, given a pair of racy
%% instructions, this technique flags  the race as benign if (1) for a
%% write-write race, both instructions write the same value or (2) for a
%% read-write race, the value written by one instruction does not affect the
%% value read by another.

%% % Researchers have found that conventional races outnumber schedule races
%% % by orders of magnitude~\cite{pres:sosp09}.

%% Our detector also flags \emph{symbolic races} that involve symbolic memory
%% locations.  Figure~\ref{fig:symbolic-race} shows an example.  Both
%% variables $i$ and $j$ are inputs, and the race occurs only when $i = j$.
%% The risk of such symbolic races is that they may be absent in memoization
%% runs and skip detection, but show up nondeterministically in reuse runs.

%% To detect symbolic races, our race detector queries the underlying
%% symbolic execution engine for pointer equality.  If a symbolic race is
%% possible, we add additional input constraints to prevent the race from
%% occurring in a reuse run.  For instance, for the race in
%% Figure~\ref{fig:symbolic-race}, we will add $i \neq j$ to the input
%% constraints of the memoized schedule.  
%% % Our race detector is sound if the \v{symbolic()} annotations capture all
%% % nondeterminism that may alter the memory locations accessed.


\subsection{Skipping Unnecessary Synchronizations}  \label{sec:tern-skip-waits}

When reusing a schedule, \tern enforces a total synchronization order
according to the schedule.  These \tern-enforced execution order constraints
are more stringent than the constraints enforced by the original
synchronizations in the program.  Thus, for speed, \tern can actually skip
these unnecessary synchronizations.  In our current implementation, we
skip \v{sleep()}, \v{usleep()}, and \v{pthread\_barrier\_wait()}
because they are frequently used.  We
found that this optimization was quite effective and
even made programs run faster than nondeterministic execution
(\S\ref{sec:tern-overhead}).

% For instance, we can skip the actual \v{pthread\_mutex\_lock()} in
% Figure~\ref{fig:replayer} as well as other pthread lock-related
% functions.

%% One risk is that the original program may break abstraction boundaries and
%% peek into the internals of the synchronization objects, while \tern has
%% skipped the corresponding synchronization operations.  


% this problem is analog to double paging.

% \tern enforces order.   barrier_wait is redundant. but cause a sleep.

\subsection{Simplifying Constraints} \label{sec:tern-simplify}

To reuse a schedule, \tern must check if the current input satisfies the
constraints of the schedule.  The overhead of this check depends on the
number of constraints, yet the set of constraints \tern collects may not
always be in simplified form.  That is, a subset of the constraints may
imply the entire set.  For example, consider a loop ``\v{for(int
  i=0;i!=n;++i)}'' with a symbolic bound $n$.  When running this code with
$n=10$, we will collect a set of constraints $\{0 \neq n, 1 \neq n, ...,
10 = n\}$, but the last constraint alone implies the entire set.

To simplify constraints, \tern uses a greedy algorithm.  Given a set of
constraints $C$, it iterates through each constraint $c$, and checks if
$C/\{c\}$ implies $\{c\}$.  If so, it simply discards $c$.  Our
observation is that constraints collected later in a run tend to be more
compact than the earlier ones.  Thus, when pruning constraints, we start
from the ones collected earlier.  Although we could have used the
underlying symbolic execution engine to simplify constraints, it lacks
this domain knowledge and may perform poorly.

\subsection{Slicing Out Irrelevant Branches} \label{sec:tern-slicing}

A branch statement may observe a piece of symbolic data but perform no
synchronization operation in either branch.  The constraints collected
from this branch are unlikely to affect schedules.  If we include
irrelevant constraints in the input constraints of a schedule, we not only
increase constraint checking time, but also preclude legal reuses of the
schedule.

To address this problem, \tern employs a simple static analysis to
automatically prune likely irrelevant constraints.  At the heart of this
technique is a slicing analysis that identifies branch statements unlikely
to affect synchronization operations.  Specifically, given a branch
statement $s$, this analysis computes $s_d$, the immediate
post-dominator~\cite{aho:dragon:06} of $s$, and marks $s$ as irrelevant if
no synchronization operations are between $s$ and $s_d$.  Although simple,
this technique reduced constraint checking time significantly
(\S\ref{sec:tern-overhead}).  However, we note that our analysis is unsound
because it ignores data dependencies.  Thus, we plan to implement a sound
slicing algorithm~\cite{castro:bouncer} in our future work.



%% \subsection{Memoizing Schedules Online}

%% online version:
%%    batch  - rerun

%%    server - forward

%%       haven't implemented forwarding for MySQL

%% By default, \tern works in offline mode: 

%% a developer pre-populates a schedule cache and hardwires it to her
%% application.


%% memoizes schedules offline.  

%% default: offline.  already very useful.  evaluation results.

%% hower, to react better to dynamic load: online.


%% We implemented online memoization differently for batch and server
%% programs.  For batch programs, we used a restart approach: if \tern
%% cannot find a memoized schedule for an input, it restarts the 

%% To avoid interfering the actual run, \tern sandboxes the shadow run by
%% crudely forking a process and dropping all file and network
%% output. (Better sandboxes can of course be built.)

%%   Specifically,
%% at program start, \tern records the command line arguments.  In one of the
%% \v{symbolic()} calls, if \tern cannot find a schedule
%% for the finds out that the current input satisfies
%% no memoized constraints, it switches to the memoizer and restarts the
%% program on the recorded command line arguments by calling \v{exec()}.  For
%% server programs, we used a shadow program approach because restarting
%% server tends to be costly.  Specifically, we run two versions of the
%% program, one with the replayer and the other with memoizer.  User requests
%% are sent to the replayer version of the program.  If the replayer cannot
%% find a schedule for some input, it forwards the input the memoizer.

%% For either type of programs, \tern must ensure that 

%% restarting or forwarding

%% Our current implementation provides only simple output buffering.

%% better sandbox can be built.



%% \subsection{Speeding Up Reuse Runs}

%% We used two techniques to speed up the runs that reuse schedules.  The
%% first is called \emph{relay semaphores}.  To reuse a schedule, we must
%% enforce the execution order constraints of the schedule.  We experimented
%% a variety of methods such as letting the threads spin-waits or using
%% condition variables.  We finally settled with semaphores which tend to
%% perform the best.  Specifically, we let each threads wait on a
%% thread-local semaphore before it gets its turn according to a memoized
%% schedule.  Once a threads is done with its turn, it finds the next thread
%% according to the schedule and signals the thread's semaphore; it then
%% waits for its turn again by waiting on its semaphore.  By block waiting on
%% semaphores, we avoid the overhead of spin waits; by deterministically
%% relaying turns, we avoid the lock contention overhead of a condition
%% variable solution.

%% The second is called \emph{skipped wait}.  

%% memoization run: wait

%% reuse run, do not have to wait.  

%% sleep(30)

%% do not have to wait 30 seconds.  

%%   - skip wait

%%   - more aggressive: skip sync operations, peek the sync var values.  

%%     cite R2 interface thing.  


%% The third is constraint simplification.

%% redundant constraints   0 < n 2 < n 3 >= n



%% \subsection{Assisting Developer Annotations} \label{sec:slicing}

%% The accuracy of the \v{symbolic()} annotations affect \tern's effectiveness.
%% If developer ``over-mark'' too much data as symbolic, the constraints
%% \tern collects may be too specific and prevent legitimate schedule reuses;
%% if developer ``under-mark'' too little, the constraints may be too general
%% and cause \tern to wrongly reuse schedules on incompatible inputs.

%% We created two techniques to simplify this annotation task.  Our first
%% technique automatically prunes likely irrelevant constraints, thus
%% tolerating over-marks.  For instance, with this technique, we   were able to achieve high reuse rates in our experiments
%% despite that we marked the entire HTTP requests to Apache and all command
%% line arguments to the batch programs as symbolic.  At the heart of this
%% technique is a static slicing analysis that identifies branch statements
%% unlikely to affect synchronization operations.  Specifically, given a
%% branch statement $s$, this analysis computes $s_d$, the immediate
%% post-dominator~\cite{xxx-some-compiler-book} of $s$, and marks constraints
%% from $s$ as likely irrelevant if no synchronization operations are between
%% $s$ and $s_d$.  Note our slicing analysis is unsound because it ignores
%% data dependency.  We could have used sound precondition
%% slicing~\cite{castro:bouncer}, but our simple, unsound analysis worked
%% well in our experiments.




%% \tern collects for a schedule may be over-constraining,
%% \ie, containing superfluous constraints that do not affect the schedule.
%% For instance, a program may check the value of a symbolic byte in an
%% \v{if}-statement, but perform no synchronizations or blocking system calls
%% in either branch.  We could have pruned such constraints using
%% precondition slicing~\cite{castro:bouncer}, but we have not implemented it
%% in \tern because the hit rate of the schedule cache is already high (cf
%% \S\ref{sec:evaluation}).  Moreover, \tern can improve its schedule cache
%% upon cache misses by deriving fresh schedules and merging them into the
%% cache.


%% Our second technique tolerates under-marks by storing extra schedules.

%% The intuition is that

%% When developers under mark

%% an input satisfies input constraint, but still cannot reuse the schedule.


%% Specifically, for a single set of input constraints, we may store more
%% than one schedules in the schedule cache.  We order these schedules based
%% on their reuse rates in descending order, the same strategy as schedule
%% cache replacement (\S\ref{sec:cache-replace}).  



%% remmber more than one schedules for a set of constraints.

%% if one cannot be reused, rank it lower.

%% use the same cache replacement algorithm







%% our second technique modifies the schedule cache xxx.

%% under one constraint check, multiple schedules.

%% sort them uses finish ratio, same as cache replacement.

%% a schedule frequently broken ranked lower, 

%% so that we will use frequently finish schedule first.


%% our last technique assists in annotation.  when schedule is broken,
%% position last common hook in old and new schedule.  also first unique hook
%% of the schedules.  computes the branches that (1) appeared in the run and
%% (2) 



%% path constraints too specific

%%    specific file blocks to compress does not affect interleaving

%%    if keep in constraints, 

%% simple slicing that use dominance.  if any branch dominates 



%% The planned \tern design described so far considers input constraints for a
%% particular run.  Although this method already allows us to reuse a
%% schedule across many inputs, it may still contain superficial constraints
%% that unnecessarily limit the applicability of a schedule.  For example, an
%% \v{if}-statement may inspect some input data but then perform the same
%% synchronizations and memory accesses in both the true and false branches.
%% Figure~\ref{fig:superficial}(a) shows such an example.  The constraints on
%% $i$ and $j$ are superficial because regardless of their values, the
%% synchronization order shown can always be reused deterministically.


   

%%   - cache replacement

%%   - divergence detection



%% %% \subsection{Adapting \klee}

%% %% To adapt \klee to \tern, we made four main modifications.  First, \klee works
%% %% with only sequential program, and we thus extended \klee to support threads
%% %% and shared memory.  Specifically, we run a \klee instance for each thread
%% %% in a program, and let these \klee instances store constraints in shared
%% %% memory.  Second, we simplified \klee to only collect constraints without
%% %% solving them, because unlike \klee, \tern does not need to explore different
%% %% execution paths.
%% %% % \klee is designed to
%% %% % generate many different testcases to test many paths.  The original \klee
%% %% % solves constraints to explore different execution paths of a program, and
%% %% % constraint solving can be expensive and unnecessary for the memoizer.
%% %% % It does so by
%% %% % systematically exploring both branches of a symbolic conditional, which
%% %% % involves much (slow) constraint solving.  
%% %% Third, due to a limitation of the constraint solver \klee uses, it
%% %% concretizes symbolic divisions (either the divisor or the dividend is
%% %% symbolic) when divisor is not a power of two.  We modified \klee to simply
%% %% ignore symbolic divisions.  
%% %% Lastly, we extended \klee to track both symbolic and concrete values for
%% %% symbolic inputs so that we can concretize the symbolic inputs with
%% %% meaningful values (cf \S\ref{sec:window}).

%% %% how to modify klee to allow it to run multi-threaded programs.
%% %% Need to consider "design level" of modification to klee (as a black box).
%% %% Change two level memory model to single memory model.
%% %% Add "shared memory" feature, since original klee copies the state (and each state denotes a fork, so they do not share memory), and this does not follow the threading principle.
%% %% Change klee to be a constraint collector (not just simplifying klee, but also change klee to be a concrete/symbolic mix execution engine). 
%% %% No longer solving any constraints, just check whether input satisfy constraints (input makes all constraints to be true).

%% %% Below are some implementation details which may be interesting.

%% %% Handling thread creation (how to instrument pthread create() with LLVM, and redirect the pthread create() functions and change PC of IRs).

%% %% How to handle determinisit thread creation in klee (assuming thread pool). This is crucial for maintaing my own thread ids deterministically).


%% %% How to add hook functions with LLVM.



%% %% - internal cache problem.  add more mark symbolic

%% %% - input timing.  recv().  sleep().





%% %% x/symbolic  : skip making sybmolic = 0

%% %% dbt2: make floating point 

%% %% z = y/x: division.  don't do symbolic.

%% %% modifications to programs:  

%% %%     - uninitialized reads

%% %%     - 


%% %% dbt2 floating point
