\section{Background}
\label{sec:background}

This section presents a analytical background of \tern. We first explain the instability
problem of existing \dmt systems (\S\ref{sec:dmt-background}), and then our choice
of schedule representation in \tern (\S\ref{sec:define-schedule}).

\subsection{The Instability Problem in \dmt} \label{sec:dmt-background}

A \dmt system is, conceptually, a function that maps an input $I$ to a
schedule $S$.  The properties of this function are that the same $I$
should map to the same $S$ and that $S$ is a feasible schedule for
processing $I$.  A stable \dmt system such as \tern has an additional
property: it maps similar inputs to the same schedule.  Existing \dmt
systems, however, tend to map similar inputs to different schedules, thus
suffering from the instability problem.  

We argue that this problem is inherent in existing \dmt systems because
they are stateless.  They must provide the same schedule for an input
across different runs, using information only from the current run.
To force threads to communicate (\eg, acquire locks or access shared memory)
deterministically, existing \dmt systems cannot rely on physical clocks.  Instead, they
maintain a logical clock per thread that ticks deterministically based on
the code this thread has run.  Moreover, threads may communicate only when
their logical clocks have deterministic values (\eg, smallest across the
logical clocks of all threads~\cite{kendo:asplos09}).  By induction,
logical clocks make threads deterministic.

However, the problem with logical clocks is that for efficiency,
they must tick at
roughly the same rate to prevent a thread with a slower clock from
starving others.  Thus, existing \dmt systems have to tie their logical
clocks to low-level instructions executed (\eg, completed
loads~\cite{kendo:asplos09}).  Consequently, a small change to the input or
execution environment may alter a few instructions executed, in turn
altering the logical clocks and subsequent thread communications.  That is,
a small change to the input or execution environment may cascade into a
much different (\eg, correct vs. buggy) schedule.  






