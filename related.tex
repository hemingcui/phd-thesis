
\chapter{Related Work} \label{sec:related}

Todo: add the limitation of \dmt here. Can refer to the CACM paper.

%%\section{\dmt and \smt systems} \label{sec:related-dmt}
\para{DMT and \smt systems.}  Conceptually, prior
work~\cite{dthreads:sosp11, cui:tern:osdi10, peregrine:sosp11,
  determinator:osdi10}, including ours~\cite{cui:tern:osdi10,
  peregrine:sosp11}, conflated determinism and stability.  To our
knowledge, we are the first to distinguish these two separate
properties~\cite{smt:cacm,smt:hotpar13}.

Implementation-wise, several prior systems are not backward-compatible
because they require new hardware~\cite{dmp:asplos09}, new
language~\cite{dpj:oopsla09}, or new programming model and
OS~\cite{determinator:osdi10}.  Among backward-compatible systems, some
DMT systems, including \kendo~\cite{kendo:asplos09},
\coredet~\cite{coredet:asplos10}, and \coredet-related
systems~\cite{dos:osdi10, ddos:asplos13}, improve performance by balancing
each thread's load with low-level instruction counts, so they are not
stable.

Five systems can be classified as \smt systems.  Our
\tern~\cite{cui:tern:osdi10} and \peregrine~\cite{peregrine:sosp11} systems
require sophisticated program analysis to determine input and schedule
compatibility, complicating deployment. Bergan {\it et
  al}~\cite{bergan:oopsla13} built upon the ideas in \tern and \peregrine
to statically compute a small set of schedules covering all inputs, an
undecidable problem in general.  \grace~\cite{grace:oopsla09} and \dthreads~\
cite{dthreads:sosp11} ignore thread load
imbalance, so they are prone to the serialization problem
($\S$\ref{sec:example}). \grace also requires
fork-join parallelism.

Compared to \parrot's evaluation, prior evaluations have several limitations.
First, prior work has reported results on a narrow set of programs,
typically less than 15.  The programs are mostly synthetic benchmarks, not
real-world programs, from incomplete suites.  Second, the experimental
setups are limited, often with one workload per program and up to 8
cores.\footnote{Three exceptions used more than 8 cores:
  \cite{kendo:wodet11} (ran a 12-line program on 48 cores),
  \cite{aviram:thesis} (ran 9 selected programs from \parsec, \splashx, and
  \npb on 32 cores), and \cite{dmp:asplos09} (emulated 16 cores).}

Lastly, little prior work except ours~\cite{wu:pldi12} has demonstrated how the
approaches benefit testing or reported any quantitative results on
improving reliability, making it difficult for potential adopters to
justify the overhead.

\para{State-space reduction.} \parrot greatly reduces the state space of
model checking, so it bears similarity to \emph{state-space reduction
  techniques} (\eg,~\cite{godefroid:verisoft, flanagan:dynamicpo,
  demeter:sosp11}).  Partial order
reduction~\cite{godefroid:verisoft, flanagan:dynamicpo} has been the main
reduction technique for model checkers that check implementations
directly~\cite{modist:nsdi09, dbug:spin11}.  It detects
permutations of independent events, and checks only one permutation
because all should lead to the same behavior.  Recently, we proposed
dynamic interface reduction~\cite{demeter:sosp11} that checks loosely
coupled components separately, avoiding expensive global exploration of
all components.  However, this technique has yet to be shown to work well
for tightly coupled components such as threads communicating
via synchronizations and shared memory.
% (The explorer of each component in~\cite{demeter:sosp11} has to explore
% many different schedules, which \parrot can help reduce.)

\parrot offers three advantages over reduction techniques
(\S\ref{sec:mc}): (1) it is much simpler because it does not
need equivalence to reduce state space; (2) it remains effective as the
checked system scales; and (3) it works transparently to reduction
techniques, so it can be combined with them for further reduction.  The
disadvantage is that \parrot has runtime overhead.

\para{Concurrency.} Automatic mutual exclusion (AME)~\cite{ame:hotos07} 
assumes all shared
memory is implicitly protected and allows advanced developers the
flexibility to remove protection.  It thus shares a similar high-level
philosophy with \parrot, but the differences are obvious.  We are unaware of
any publication describing a fully implemented AME system. \parrot is
orthogonal to much prior work on concurrency error
detection~\cite{yu:racetrack:sosp,savage:eraser,racerx:sosp03,lu:muvi:sosp,
avio:asplos06,conmem:asplos10},
diagnosis~\cite{racefuzzer:pldi08,ctrigger:asplos09,atomfuzzer:fse08}, and
correction~\cite{dimmunix:osdi08,gadara:osdi08,wu:loom:osdi10,cfix:osdi12}. By
reducing schedules, it potentially benefits all these techniques.


