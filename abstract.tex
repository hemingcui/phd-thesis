Multithreaded programs have become pervasive and critical due to the rise of the
multi-core hardware and the accelerating computational demand.
Unfortunately, despite decades of research and engineering effort, these
programs remain notoriously difficult to get right, and they are plagued with
harmful concurrency bugs that can cause wrong outputs, program crashes, security
breaches, and so on.

Our research reveals that a root cause of this difficulty is
that multithreaded programs have too many possible thread interleavings (or
\emph{schedules}) at runtime. Even given only a single input, a program may run 
into a great number of schedules, depending on factors such as hardware timing 
and OS scheduling. Considering all inputs, the number of schedules is even much 
greater. It is extremely challenging to understand, test, analyze, or verify 
this huge number of schedules for a multithreaded program and make sure that 
all these schedules are free of concurrency bugs. Thus, multithreaded programs 
are extremely difficult to get right.

To reduce the number of possible schedules for all inputs, we looked into the 
relation between inputs and schedules of real-world programs, and made an 
exciting discovery: many programs need only a small set of schedules to 
efficiently process a wide range of inputs! Leveraging this discovery, we have 
proposed a new idea called \emph{Stable Multithreading} (or \emph{\smt}) that 
reuses each schedule on a wide range of inputs, greatly reducing the number of 
possible schedules for all inputs. By addressing the root cause that makes 
multithreading difficult to get right, \smt makes understanding, testing, 
analyzing, and verification of multithreaded programs much easier.

To realize \smt, we have built three \smt systems, \tern, \peregrine, and 
\parrot, with each addressing a distinct research challenge. Evaluation on a 
wide range of 108 popular multithreaded programs with our latest \smt system, 
\parrot, shows that \smt is simple, fast, and deployable. All \parrot's source 
code, entire benchmarks, and raw evaluation results are available at \github.

To encourage deployment, we have applied \smt to improve several reliability 
techniques, including: (1) making reproducing real-world concurrency bugs much 
easier;  (2) greatly improving the precision of static program analysis, 
leading to the detection of several new harmful data races in heavily-tested 
programs; and (3) greatly increasing the coverage of model checking, a 
systematic testing technique, by many orders of magnitudes. \smt has attracted 
the research community's interests, and some techniques and ideas in our \smt 
systems have been leveraged by other researchers to compute a small set of 
schedules to cover all or most inputs for multithreaded programs.

% All these advances on building and applying \smt has made it an easy to use, 
% fast, and much more reliable and secure multithreading approach, potentially 
% benefiting all individuals, governments, organizations, and software vendors. 
% 
