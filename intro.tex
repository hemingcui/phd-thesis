\section{Introduction} \label{ch:intro}

%% P1: Multi-threading is critical and popular.
Multi-threading has become increasing pervasive and critical because of two 
trends. First, due to physical constraints on circuit speed, computers are having
more and more cores rather than faster and faster single-core. In order to 
harness the power of multi-core, more and more software are multi-threaded. 
Second, the emerging cloud computing trend requires web services, ranging from 
web servers to databases, to process more and more requests at the same time, which also 
pushes developers to write more and more multi-threaded programs in order to 
process a large amount of requests concucurrenlty. These trends continue and 
then multit-threading are becoming increaingly pervasive and critical.

%% P2: However, it is hard to get right.
Unfortunately, despite much efforts from both academia and industry, 
multi-threaded programs are still notoriously difficult to get right. A key 
reason is: even running on the same input, the concurrently running threads of 
a program may interleave in \v{too many} different ways, and we call these interleavings 
\v{schedules}. Considering all inputs, this number is even larger. 
It is extremely hard to test, analyze, and verify all these huge 
amount of possible schedules and make sure these programs are reliable to 
concurrency bugs. As a result, a concurency bug hidden in one schedule can 
simply bypass these reliability techniques and show up in the field, 
potentially causing program crashes, wrong outputs, security breaches, and so on.

%% P3: DMT: one possible approach. And its limitations.
In order to reduce the number possible schedules on each input, researchers 
have proposed a great idea called deterministic multithreading (or \dmt) that 
always enforces the same schedule on the same input. By mapping each input to onely 
one schedule, \dmt greatly improved reliability for multithreaded programs on each input.

%% P4: StableMT: our new approach and thesis.
However, although \dmt is useful, it is not as useful as commonly perceived, 
because as we have shown in our previous research~cite{tern:osdi10},
a typical \dmt system can map slightly different inputs to 
very different schedules, artificially reducing programs' robustness on input 
perturbation, and the number of possible schedules on all inputs are still too 
many.

%% P5: StableMT's applications.

%% P6: Structures of this proposal.

%% Multithreading has become one main stream of computer software due to the rise 
%% of the multicore and the trend of big data.

%% Unfortunately, despite much effort, multithreading is still extremely hard to 
%% get right, and a key reason is programs may run into too many possible thread 
%% interleavings (or schedules) at runtime. This excessive number of schedules 
%% greatly aggravate understanding, testing, analyzing of software. Refer to XXX: 
%% one grand challenge.

%% In order to mitage this problem, researchers have proposed a great idea call 
%% \dmt. However, these systems aim at reducing the number of possible schedules 
%% on each input. And these sytems may map similar inputs into very different 
%% schedules, artificially reducing programs' robustness. Conserding all inputs, 
%% the number of possible schedules can still be too many, and the resultant 
%% programs are still hard to understand, test, analyze, and so on.

%% In order to reduce the number of possible schedules for all inputs, my collaborators and I have 
%% proposed a new idea called Stable Multithreading (or \smt) that can greatly 
%% reduce the number of possible schedules for all inputs, greatly benefitting 
%% understanding of multithreaded programs as well as almost all reliability tools.

%% Our previous reserach, three softare systems, with each addressing different 
%% challenges.

\tern TBD.

\peregrine TBD.

\parrot TBD.

Our \smt systems have broad applications.

Applying \smt to improving model checking coverage.

Applying \smt to build transparent multithreaded state machine replication systems.

The rest of the article is organized as follows.

