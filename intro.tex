\section{Introduction} \label{ch:intro}

Multithreading has become one main stream of computer software due to the rise 
of the multicore and the trend of big data.

Unfortunately, despite much effort, multithreading is still extremely hard to 
get right, and a key reason is programs may run into too many possible thread 
interleavings (or schedules) at runtime. This excessive number of schedules 
greatly aggravate understanding, testing, analyzing of software. Refer to XXX: 
one grand challenge.

In order to mitage this problem, researchers have proposed a great idea call 
\dmt. However, these systems aim at reducing the number of possible schedules 
on each input. And these sytems may map similar inputs into very different 
schedules, artificially reducing programs' robustness. Conserding all inputs, 
the number of possible schedules can still be too many, and the resultant 
programs are still hard to understand, test, analyze, and so on.

In order to reduce the number of possible schedules for all inputs, my collaborators and I have 
proposed a new idea called Stable Multithreading (or \smt) that can greatly 
reduce the number of possible schedules for all inputs, greatly benefitting 
understanding of multithreaded programs as well as almost all reliability tools.

Our previous reserach, three softare systems, with each addressing different 
challenges.

\tern TBD.

\peregrine TBD.

\parrot TBD.

Our \smt systems have broad applications.

Applying \smt to improving model checking coverage.

Applying \smt to build transparent multithreaded state machine replication systems.

The rest of the article is organized as follows.

