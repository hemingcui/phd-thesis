
\section{Related Work} \label{sec:crane-related}

This chapter proposes \crane, the first practical multithreaded SMR system
design. Our key weapon is Stable Multithreading (or \smt), a reliable
multithreading runtime technique invented by my collaborators and me. \smt can
avoid diverged thread interleavings by greatly reducing the number of possible
thread interleavings at runtime, making replications of multithreaded programs
almost as easy as single-threaded ones. \crane is related to three strands
of the literature:  SMR systems, stable and deterministic multithreading
systems, and checkpointing techniques.

%%\subsection{Related Work by Others} \label{sec:crane-others-work}

\para{State machine replication (SMR).}  SMR has been studied by the literature
for decades, and it is recognized by both industry and academia as a powerful
fault-tolerance technique in clouds and distributed systems~\cite{lamportclock,
smr:tutorial}. As a common standard, SMR uses \paxos~\cite{paxos} as the
consensue protocol to ensure that all replicas see the same input request
sequence. In this standard, replicas first ``agree" on a total order of input
request as a input sequence, and then ``execute" the requests that have reached
this consensus. This typical SMR approach is called ``agree-execute". SMR
systems, including Chubby~\cite{chubby:osdi}, ZooKeeper~\cite{zookeeper}, and
the Microsoft \paxos~\cite{paxos} implementation, have been widely used to
maintain critical distributed systems configurations (\eg, group leaders,
distributed locks, and storage meta data). SMR has also been applied broadly to
build various highly available services, including
storage~\cite{paxos:datastore} and wide-area 
network~\cite{mencius:osdi08}. These systems all require single-threading in the
replicated service.

%% Work to support multithreading.
In order to support multithreading in SMR, Eve~\cite{eve:osdi12} introduces a
new ``execute-verify" approach: it first executes a batch of requests
speculatively, and then verifies whether these requests have conflicts that
cause execution state divergence. If so, Eve rolls back the program to a state
before executing these requests and re-execute these requests sequentially. Both
Eve's execution divergence verification and rollbacks require developers to
annotate all shared states (\eg, global variables abd heaps) of a program, which
could be time consuming and error-prone. CBASE~\cite{cbase:dsn04} is similar to
Eve, but it assumes no conflicts on requests it batches.

Rex~\cite{rex:eurosys14} addresses the thread interleaving divergence problem
with a ``execute-agree-follow" approach: it first records thread interleavings
on the primary replica by executing requests, and then replays these
interleavings on the other replicas. If the executed interleavings in the
primary may not be agreed on the other nodes, then Rex will rollback the
primary's states. These rollbacks/checkpoints also require developers' manual
efforts for every program. Furthermore, Rex requires frequently shipping thread
interleavings across replicas, which may be much slower than our \smt approach
that schedules each replica indepently. NonStop Advanced
Architecture~\cite{nonstop:dsn05} is similar to Rex, but it can only do
replications on a local machine.

There are other work that support restricted concurrency such as
transactions~\cite{vandiver:sosp07, kobus:icdcs13} in SMR, but they are not for
replicating general multithreaded programs.

\para{\smt and \dmt systems.}  In order to make multithreading easier to
undertand, test, analyze, and replicate, researchers have built two types of
reliabile multithreading systems: (1) stable multithreading systems (or
\smt)~\cite{grace:oopsla09, dthreads:sosp11, determinator:osdi10} that aim to
reduce the number of possible thread interleavings for program all inputs, and
(2) deterministic multithreading systems (or \dmt)~\cite{dpj:oopsla09,
dmp:asplos09, kendo:asplos09, coredet:asplos10, dos:osdi10, ddos:asplos13,
ics:oopsla13} that aim to reduce the number of possible thread interleavings on
each program input. Typically, these systems use deterministic logical clocks
instead of nondeterministic physical clocks to make sure inter-thread
communications (\eg, \mutexlock and accesses to global variables) can only
happen at some specific logical clocks. Therefore, given the same or similar
inupts, these systems can enforce the same thread interleavings and eventually
the same executions. These systems 
have shown to greatly improve software reliability, including coverage of
testing inputs~\cite{ics:oopsla13} and speed of recording
executions\cite{dos:osdi10} for debugging.

Typical DMT systems, including \kendo~\cite{kendo:asplos09},
\coredet~\cite{coredet:asplos10}, and \coredet-related systems~\cite{dos:osdi10,
ddos:asplos13}, improve performance by balancing each thread's load with
low-level instruction counts, so they are not stable to input perturbations.
Therefore, we argue that \smt is more useful than \dmt for software reliability.

%% First introduce checkpoint techniques that do not support threads.
\para{Checkpointing program states.} Checkpointing is a common technique in
systems with critical fault-tolerance demands, and the specific checkpoint
requirements depend on the target system design. In the last few decades,
checkpointing distributed systems~\cite{beguelin:jpdc97, dejavu:ipdps07,
dmtcp:ipdps09, oren:atc07} is well studied. These systems focus on checkpointing
execution states of a distributed system consistently, and they didn't aim to
address the challenges of checkpointing multiple threads in a program.

%% Introduce techniques that support threads.
Recently, checkpointing threads on multi-core is studied in a debugging
technique called deterministic
replay~\cite{smp-revirt:vee08,pres:sosp09,odr:sosp09,scribe:sigmetrics10,
capo:asplos09}.
For instance, SMP-ReVirt~\cite{smp-revirt:vee08} and
Scribe~\cite{scribe:sigmetrics10} use clever page protection tricks to record
memory accesses. Unfortunately, these checkpoint techniques do not meet \crane's
purpose, because the replayed executions in these systems are mainly for
debugging and don't need to go live (\eg, the network sockets and file
descriptors are normally fake). However, our proposed \crane system requires
that the checkpointed executions must be live on remote replicas and can serve
future requests.

Prior multithreaded SMR systems mentioned in this section~\cite{eve:osdi12,
rex:eurosys14} provide checkpointing schemes under their systems design, but
these schemes require developers' manual efforts to annotate shared thread
states, which our \crane system expects to avoid.


A key requirement to making SMR practically support multithreading is that all
replicas must run the same thread interleavings so that the replica executions
won't diverge. \smt~\cite{smt:cacm}, an advanced reliable multithreading
runtime technique invented by my collaborators and me, meets this requirement.
This is because \smt can greatly reduce the number of possible thread
interleavings in multithreaded programs for all inputs with low performance
overhead, making replications of multithreaded programs almost as easy as
single-threaded ones. In the last four years, we have been collaborating with
Columbia and CMU researchers to build three \smt systems,
\tern~\cite{cui:tern:osdi10}, \peregrine~\cite{peregrine:sosp11}, and
\parrot~\cite{parrot:sosp13}, with each addressing distinct research challenges.
Notably, \parrot, our latest system, is the first \smt runtime system that is
fast (12.7\% mean overhead for all evaluated programs) on a wide range of 100+
popular multithreaded programs. We have 
put all \parrot's source code and raw evaluation results on
\url{http://github.com/columbia/smt-mc} for future research and industrial
deployments. Due to \parrot's high practicality, we plan to leverage it in this
proposed \crane system.

To show \smt's potential, we have applied these systems to greatly improving
software reliability and security, including improving precision and simplicity
of program analysis and verification~\cite{wu:pldi12}, making debugging
concurrency errors much easier~\cite{cui:tern:osdi10}, and improving coverage of
model checking~\cite{parrot:sosp13}. Our work have also been leveraged by the
community: some techniques in our \tern system~\cite{cui:tern:osdi10} has been
used by University of Washington Seatle researchers on computing a small set of
thread interleavings covering all inputs~\cite{ics:oopsla13}, and our \parrot
runtime system has been integrated with a CMU software model
checker~\cite{dbug:spin11}.

% %%Our experience on building \smt runtime sytems can address the two
% aforementioned major research challanges of the \smt and SMR integration. In
% addition, our techniques on program analysis, verification, and model checking
% can be deployed in \crane replicas and enhance their reliability and security.


