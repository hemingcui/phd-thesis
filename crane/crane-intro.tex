This chapter presents \crane, the first practical system design that supports
multithreading for state machine replication (or \smr), a critical
fault-tolerance technique in today's clouds and distributed systems.

\section{Introduction} \label{sec:rep-intro}

A typical \smr system runs the same program on a
number of server machines (or replicas),
and makes all replicas reach consensus on the same sequence of input requests.
Because replicas perform the same execution state transitions on the input
sequence and their executions never diverge, SMR makes a program highly
fault-tolerant for cloud providers and users, as long as more than half of
replicas still run and communicate normally. Unfortunately, as more and more
programs shift
from single-threaded to multithreaded in order to leverage multi-core speed,
SMR can easily break: the concurrently running threads within a program may
interleave in too many different ways at runtime, then replicas can easily run
into divergent thread interleavings and have divergent execution states. To
support multithreading in SMR, 
existing research work require developers to manually annotate shared thread
states in each program, which could be time consuming, error-prone, and thus
impractical.

% This project aims to build the first practical multi-threaded SMR system
%5called
% \crane. Our key to avoiding manual annotation efforts is Stable
%Multi-Threading
% (or StableMT), a runtime technique that can avoid divergent executions by
% greatly reducing the number of possible thread interleavings at runtime.
% However, although the outcome of integrating StableMT with SMR is promising,
% there are two major research challenges. First, in different replicas, each
% request may arrive and be processed in different physical timings by StableMT,
% causing divergent thread interleavings. One may consider to assign each
%request
% a time value and let replicas reach consensus on this value, but as analyzed
%in
% the following sections, this approach can cause significant slowdown or even
% StableMT deadlocks. To address this challenge, we propose a new algorithm that
% efficiently enforces consistent time values across replica without requiring
% consensus. Second, to avoid execution divergence on extreme cases such as
% machine failures, 
% \crane requires an efficient program checkpoint scheme for application-level
% states, including memory, threads, file systems, and network sockets. However,
% to our best knowledge, a suitable solution does not exist. We propose to
% leverage the latest advances in hardware-accelerated virtualization to track
% memory states efficiently, and we propose a careful design of the \crane
% architecture to systematically checkpoint threads, file systems, and sockets.

% This project aims to make state machine replication (or SMR) support
% multi-threading, so that cloud providers, developers, and users can enjoy
% highly available services as well as fast performance on multi-core hardware.
% SMR systems, including the Google Chubby lock service, the Microsoft MPS Paxos
% library, and the Apache ZooKeeper coordination service, is a fault-tolerance
% technique widely used in today's clouds and distributed systems. By running
% the same program on replicas, and enforcing a consistent input sequence for
% all these replicas, SMR can enforce all replicas to perform the same execution
% state transitions and never diverge. Unfortunately, as more and more programs
% shift from single-threaded to multi-threaded in order to leverage multi-core
% speed, existing thread runtimes allow too many possible thread interleavings,
% which can cause different replicas to easily run into diverge interleavings
% and execution states. This problem has isolated the multi-core performance
% benefits from the SMR's 
% availability and reliability benefits. To address this problem, several work
% have been published in recent years' top systems conferences. However, these
% work require developers to manually annotate shared states in a program, which
% is time consuming, error-prone, and thus impractical.

This chapter proposes \crane, the first practical multithreaded SMR system
design.
\crane avoids manual annotation efforts by leveraging Stable
Multithreading (or StableMT), a reliable runtime technique invented by my
collaborators and me. By greatly reducing the number of possible thread
interleavings at runtime with low performance overhead, StableMT automatically
enforces the same thread interleavings among replicas on the same request and
avoids
execution divergence. However, although the outcome of integrating StableMT with
SMR is promising,
there are two major research challenges. First, in different replicas, each
request may arrive and be processed in different physical timings by StableMT,
causing divergent thread interleavings. One may consider to assign each request
a time value and let replicas reach consensus on this value, but as analyzed in
the following sections, this approach can cause significant slowdown or even
StableMT deadlocks. To address this challenge, we propose a new algorithm that
efficiently enforces consistent time values across replica without requiring
consensus. Second, to avoid execution divergence on extreme cases such as
machine failures, 
\crane requires an efficient program checkpoint scheme for application-level
states, including memory, threads, file systems, and network sockets. However,
to the best of our best knowledge, a suitable solution does not exist. We
propose to leverage the latest advances in hardware-accelerated virtualization
to track memory states efficiently, and we propose a careful design of the
\crane architecture to systematically checkpoint threads, file systems, and sockets.

We expect \crane to make today's clouds and distributed systems enjoy both
fast multi-core performance and high availability and reliability, potentially
benefiting all cloud providers, developers, and users.

We further expect \crane to greatly advance other reliability and security
analysis tools by the native replication feature in \crane. Existing analysis
tools (\eg, runtime concurrency error detectors and runtime secure data flow
trackers) that enhance reliability and security often introduce significant
slowdown at runtime (e.g., about 30$x$ for the Google \threadsanitizer), making
these tools hard to deploy. Fortunately, with \crane, we can deploy such
analysis tools in only one replica (we call this replica ``analyzer"), then the
other replicas (we call them ``workers") can still process input requests
rapidly, because these workers form a majority and they can reach consensus
almost synchronously. By ensuring the same input sequence and thread
interleavings for all replicas, all the reliability and security benefits in the
analyzer will also hold for the workers. In a word, \crane can mostly mask
performance slowdown for analysis tools by its native replication architecture,
and make these tools more adoptable.

The remaining of this chapter first proposes \crane's system
design (\S\ref{sec:crane-plan}), and then introduces related
work (\S\ref{sec:crane-related}).
