\chapter{Conclusion} \label{sec:conclusion}

Mulithreading is notoriously difficult to get right, and our research reveals
that a key reason is: a multithreaded program may run into exponentially many
possible schedules for all inputs at runtime, which brings a series of
significant reliability and security challenges on understanding,
testing, debugging, analyzing, verification, and replication of multithreaded
programs.

To reduce the number of possible schedules and make multithreaded
programs easier to get right, we have invented a new idea called Stable
Multithreading (or \smt) that reuses each schedule on a wide range of inputs,
greatly reducing the number of possible schedules for all inputs. Through
building three \smt systems, \tern, \peregrine, and \parrot, with each addresing
a distinct research challenge, we have shown that \smt can be made simple, fast,
and deployable. Through applying \smt to make reproducing concurrency bugs
easier, to improve precision of static program analysis, and to increase
coverage of model checking tools, we have quantitatively shown that \smt can
make multithreaded programs much easier to get right. All the source code,
benchmarks, and raw evaluation results of our latest \smt system \parrot is
available at: \github. In addition to our effort on building and applying \smt
systems, ome techniques and ideas in our \tern and \peregrine systems have been
leveraged by University of Washington researchers to compute a small set of
schedules to cover all inputs of multithreaded programs.

By addressing the key reaon that makes multithreading difficult to get right,
\smt has broad applications. In the future, we plan to apply \smt to make
replication and verification of multitheraded programs easier, and to defend
against security vulnerabilities that leverage concurrency bugs. We believe that
all these combined effort will make \smt a simple, fast, reliabile, and secure
multithreading runtime, potentially benfiting all individuals, governments,
organizations, and software vendors. 