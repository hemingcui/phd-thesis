\chapter{Conclusion} \label{sec:conclusion}

Multithreading is notoriously difficult to get right, and a root cause is that 
a multithreaded program may run into exponentially many
possible schedules for all inputs at runtime, which brings a series of
significant reliability and security challenges on understanding,
testing, debugging, analyzing, and verification of multithreaded
programs.

To make multithreading easier to get right, we have invented a new idea called 
\smt that reuses each schedule on a wide range of inputs,
greatly reducing the number of possible schedules for all inputs. Through
building three \smt systems, \tern, \peregrine, and \parrot, with each 
addressing a distinct research challenge, we have shown that \smt is 
simple, fast, and deployable. Through applying \smt to make reproducing 
concurrency bugs easier, to improve the precision of static program analysis, 
and to increase the coverage of model checking tools, we have quantitatively 
demonstrated \smt's advantages on improving software reliability. \smt has 
attracted the research community's interests, and some techniques and ideas in 
our previous systems have been leveraged by University of Washington 
researchers to compute a small set of schedules to cover all or most inputs of 
multithreaded programs. All the source code, benchmarks, and raw evaluation 
results of \parrot, our latest \smt system, are available at \github.

By addressing the root cause that makes multithreading difficult to get right,
\smt has broad applications on software reliability and security. In the 
future, we plan to apply \smt to make replication and verification of 
multithreaded programs easier, and to defend against security attacks 
that leverage concurrency bugs.