\section{Leveraging \smt to Build Transparent Multithreaded State Machine Replication} \label{sec:replication}

Due to the trends of cloud computing and multi-core hardware, building higly-
available and fast applications have become critical. In order to 
provide high availability, the state-machine replication approach has become 
popular. In order to build fast services, people write multithreaded programs 
for better throughput and response time. Todo: cite related work.

Unfortunately, despite much effort from academia and industry, building state-
machine replications for general multithreaded programs are still an open 
problem. Todo: briefly mention related work.

There are at least two major challenges towards building practical state-
machine replications for multithreaded programs. First, how to reach 
consensus on general network inputs (including input values and timings) 
across replicas. This is challenging because, in typical state-machine 
replications that are formed by primary replicas, some replicas may have 
faster state transitions than the primary. Second, how to perform read-only 
optimization for general applications, which is hard, because some 
semantically read only operations (such as get() in key-value store and GET 
in http servers) may modify states of server programs. Todo: should we 
include any other challenges that we have discussed? E.g., recovery, 
leveraging StableMT, and logical clocks.

Our system is the first system on building practical state-machine 
replications for general multi-threaded programs. Todo: how it works: reach 
consensus on general socket operations, LD\_PRELOAD, transparent (do not need 
to annotate shared states in a program) replications for servers.

Introduce the fault-tolerance and performance features (guarantees) of our 
system.

Our system made two major research contribution to address the above two 
challenges. First, we propose a epoch-based algorithm. The key insight is we 
could leverage the synchronization property in logical clocks among different 
replicas to perform consensus in a typical asynchronous network setting. Todo
: briefly mention the epoch-based algorithm. The epoch could solve the 
problem of replicas runnning ahead of primary. And it also has performance 
benefit because we no longer need to reach consensus on each socket operation 
such as recv(). Second, we introduce a new mechanism with probing program 
state changes (by hashing the schedule of each request) and fast roll back. 
This mechanism could even enable our system to run significantly faster than 
the un-replicated single node program.

Practical highlights of our system: evaluated a good range of popular server 
programs, ranging from web servers, data base servers, key-value stores, and 
popular utility programs. Todo: performance highlights, recovery highlights.
