\section{Improving Schedule Coverage of Model Checking} \label{sec:mc}


Model checking is a formal verification technique that systematically
explores possible executions of a program for bugs.  These executions
together form a \emph{state space} graph, where states are snapshots of the
running program and edges are nondeterministic events that move the
execution from one state to another.  This state space is typically very
large, impossible to completely explore---the so-called \emph{state-space explosion} problem.
To mitigate this problem, researchers have created
many heuristics~\cite{yang:fisc:osdi,musuvathi:aodv,killian:macemc:nsdi07} to 
guide the exploration toward executions
deemed more interesting, but heuristics have a risk of missing bugs.
\emph{State-space reduction} techniques~\cite{flanagan:dynamicpo,godefroid:verisoft,demeter:sosp11}
soundly prune executions without missing bugs, but the effectiveness of these techniques
is limited.  They work by discovering equivalence: given
that execution $e_1$ is correct if and only if $e_2$ is, we need check only
one of them. Unfortunately, equivalence is rare and extremely challenging
to find, especially for \emph{implementation-level} model checkers which
check implementations directly~\cite{godefroid:verisoft,musuvathi:aodv,yang:fisc:osdi,
yang:explode:osdi,killian:macemc:nsdi07,dbug:spin11}.
This difficulty is reflected in the existence of only two main reduction
techniques~\cite{flanagan:dynamicpo, demeter:sosp11} for these implementation-
level model checkers.  Moreover, as a checked system scales, the state space after
reduction still grows too large to fully explore.  Despite
decades of efforts, state-space explosion remains the bane of model
checking.

Integrating \smt and model checking is
mutually beneficial.  By reducing schedules, \smt offers an extremely
simple, effective way to mitigate and sometimes completely solve the
state-space explosion problem without requiring equivalence.  For
instance, \parrot enables \dbug to verify \nprogverifiedxxx programs,
including 4 programs containing \nondets (\S\ref{sec:coverage}).
In return, model checking helps check the schedules that matter for \parrot and developers.
For instance, it can check the default schedules chosen by \parrot, the
faster schedules developers choose using \computes, or the schedules
developers add using \nondets.

\subsection{The \dbug Model Checker} \label{sec:dbug}

In principle, \parrot can be integrated with many model checkers.  We chose
\dbug~\cite{dbug:spin11} for three reasons.  First, it is 
open source, checks implementations directly, and 
supports \pthread synchronizations and Linux 
socket calls.  Second, it implements one of the
most advanced state-space reduction techniques---dynamic partial order
reduction (DPOR)~\cite{flanagan:dynamicpo}, so the further reduction \parrot
achieves is more valuable.  Third, \dbug can estimate the size of the
state space based on the executions explored, a technique particularly
useful for estimating the reduction \parrot can achieve when
the state space explodes.

Specifically, \dbug represents the state space as an \emph{execution tree}
where nodes are states and edges are choices representing the operations
executed.  A path leading from the root to a leaf encodes a unique test
execution as a sequence of nondeterministic operations.  The total number
of such paths is the state space size.  To estimate this size based on
a set of explored paths, \dbug uses the \emph{weighted backtrack
  estimator}~\cite{Kilby2006}, an online variant of Knuth's offline
technique for tree size estimation~\cite{Knuth1975}.  It treats the set of
explored paths as a sample of all paths assuming uniform distribution over
edges, and computes the state space size as the number of explored
paths divided by the aggregated probability they are explored.

\subsection{Integrating \parrot and \dbug} \label{sec:smt+mc}

A key integration challenge is that both \parrot and \dbug control the order
of nondeterministic operations and may interfere, causing
difficult-to-diagnose false positives.  A na\"{i}ve solution
is to replicate \parrot's scheduling algorithm inside \dbug.  This
approach is not only labor-intensive, but also risky because the
replicated algorithm may diverge from the real one, deviating the checked
schedules from the actual ones.

Fortunately, the integration is greatly simplified because performance
critical sections make nondeterminism explicit, and \dbug can ignore
operations that \parrot runs deterministically.  \parrot's strong-isolation
semantics further prevent interference
between \parrot and \dbug.  Our integration uses a nested-scheduler
architecture similar to Figure~\ref{fig:parrot-arch} except the
%% Heming: TBD, add the parrot-arch figure in the parrot section.
nondeterministic scheduler is \dbug.  This architecture is transparent
to \dbug, and requires only minor changes (\locsmcmc lines) to \parrot.
First, we modified \v{nondet\_begin} and \v{nondet\_end} to turn \dbug
on and off.  Second, since \dbug explores event orders only after
it has received the full set of concurrent events, we modified
\parrot to notify \dbug when a thread transitions between the run queue
and the wait queue in \parrot. These notifications help \dbug accurately
determine when all threads in the system are waiting for \dbug to make
a scheduling decision.

We found two pleasant surprises in the
integration.  First, \computes speed up \dbug executions.  Second,
\parrot's deterministic timeout~\cite{parrot:sosp13} prevents \dbug from
possibly having to explore infinitely many schedules.  Consider the
``\v{while(!done) sleep(30);}'' loop which can normally
nondeterministically repeat any number of times before making
progress.  This code has only one schedule in the integratied system because \parrot
makes the \v{sleep} return deterministically.

\subsection{Evaluation} \label{sec:coverage}

In this evaluation, we use the same set of programs as that in \parrot, and we 
focus on how much \parrot improves \dbug's coverage.

To evaluate coverage, we used small workloads and two threads per workload.
Otherwise, the time and space overhead of \dbug,
or model checking in general, becomes prohibitive. Consequently, \parrot's
reduction measured with small state spaces is a conservative estimate of
its potential.  Two programs, \volrend and \ua, were excluded because they
have too many synchronization operations (\eg, 132M for \ua), causing
\dbug to run out of memory.  Since model checking requires a closed
(no-input) system, we paired \aget with lightweight web server
\mongoose~\cite{mongoose}).  We enabled
state-of-the-art DPOR~\cite{flanagan:dynamicpo} to evaluate how much more
\parrot can reduce the state space. We checked each program for a maximum of
one day or until the checking session finished.  We then compared the
estimated state space sizes.

\begin{table}[t]
\footnotesize
\centering
\begin{tabular}{ccl}
{\bf Bin } & {\bf \# of Programs} & {\bf State Space Size with \dbug} \\
\hline
A & 27 & $1$ $\sim$ $14$ \\
B & 18 & $28$ $\sim$ $47,330$ \\
C & 25 & $3.99\times10^{6}$ $\sim$ $1.06\times10^{473}$ \\
D & 25 & $4.75\times10^{511}$ $\sim$ $2.10\times10^{19734}$ \\
%E & 1 & N/A \\
\end{tabular}
\vspace{-.05in}
\caption{{\em Estimated \dbug's state space sizes on programs with no
    \nondet nor network operation.}  } \label{tab:state-space-compute}
\vspace{-.05in}
\end{table}

\begin{table}[b]
\footnotesize
\centering
\vspace{-.05in}
\begin{tabular}{lrrr}
{\bf Program} & {\bf \dbug} & {\bf \ecosys} & {\bf Time} \\
\hline\\[-2.3ex]
%% redis smt+mc results not out yet.
%% redis                                    & $1.40\times10^{178}$    & N/A   & No       \\
\openldap                       & $2.40\times10^{2795}$        & $5.70\times10^{1048}$    & No        \\
\redis                       & $1.26\times10^{8}$        & $9.11\times10^{7}$    & No        \\
\pfscan                                   & $2.43\times10^{2117}$     & $32,268$   & $1,201$s     \\
\aget                       & $2.05\times10^{17}$        & $5.11\times10^{10}$    & No        \\
\nthelement                        & $1.35\times10^{7}$     & $8,224$   & $309$s      \\
\partialsort                       & $1.37\times10^{7}$     & $8,194$   & $307$s      \\
\partition                           & $1.37\times10^{7}$     & $8,194$   & $307$s      \\
\fluidanimate                     & $2.72\times10^{218}$        & $2.64\times10^{218}$  & No        \\
\cholesky                       & $1.81\times10^{371}$      & $5.99\times10^{152}$   & No   \\
\fmm                           & $1.25\times10^{78}$       & $2.14\times10^{54}$   & No        \\
\raytrace                       & $1.08\times10^{13863}$        &  $3.68\times10^{13755}$   & No        \\
%% UA smt+mc results not out yet.
%%\ua                                  & $>$ \v{DBL\_MAX}        & $>$ \v{DBL\_MAX}   & No        \\
\end{tabular}
\vspace{-.05in}
\caption{{\em Estimated state space sizes for programs containing
    \nondets.}  \ecosys finished 4 real-world programs (time in
  last column), and \dbug none.} \label{tab:state-space-nondet}
\end{table}


Table~\ref{tab:state-space-compute} bins all 95 programs that contain
(1) no network operations and (2) either no hints or only \computes. For each 
program,
\ecosys reduced the state space down to just one
schedule and finished in 2 seconds. \dbug alone could finish only
\nprogverifieddbug (out of 45 in bin A and B) within the time limit.
% The reduction for the bins C and D ranges from \shrinkscale.

Table~\ref{tab:state-space-nondet} shows the results for all
\nprognondetandnetwork programs containing network operations or
\nondets.  For all four real-world programs \pfscan, \partition,
\nthelement, and \partialsort, \ecosys effectively explored all
schedules in seven hours or less, providing a strong reliability
guarantee.  These results also demonstrate the power of \parrot:
the programs can use the checked schedules at runtime for speed.

To summarize, \parrot reduced the state space by \shrinkscale for
\nprogshrink programs (50 programs in Table~\ref{tab:state-space-compute},
6 in Table~\ref{tab:state-space-nondet}).  It increased the number of
verified programs from \nprogverifieddbug to \nprogverifiedxxx (95
programs in Table~\ref{tab:state-space-compute}, 4 in
Table~\ref{tab:state-space-nondet}).

