The last chapter presents \tern, the first \smt and \dmt system. However, \tern
enforces best-effort determinism, and its executions may be nondeterministic
when data races exist. To overcome this limitation, this chapter presents
\peregrine, our second \smt and \dmt system that can still frequently reuse
schedules on many inputs, and can effoce fully determinism with reasonable
overhead.

\section{Introduction} \label{sec:intro}

%% Why multithreading is so difficult to get right. Too many schedules.
%% All inputs, each input (define nondeterminism). Must refer to Chapter 2: motivation.

%% Two complementary techniques have been proposed to address this difficulty.
%% SMT and DMT. However, it is still an open challenge to enforce a highly reusable and fully deterministic schedules efficiently.

%% Why hard. Introduce two forms of schedules
%%(1) synch schedule can be made stable and efficient, but not fully determnistic.
%%(2) Mem schedule can be made fully determinsitic, but not stable and efficient.

%% Our core technique. Schedule relaxation.
%%And why it can be stable, deterministic, and efficient.


Different runs of a multithreaded program may show different behaviors,
depending on how the threads interleave.  This \emph{nondeterminism} makes
it difficult to write, test, and debug multithreaded programs.  For
instance, testing becomes less assuring because the schedules tested may
not be the ones run in the field.  Similarly, debugging can be a nightmare
because developers may have to reproduce the exact buggy schedules.  These
difficulties have resulted in many ``heisenbugs'' in widespread
multithreaded programs~\cite{lu:concurrency-bugs}.

Recently, researchers have pioneered a technique called
\emph{Deterministic Multithreading (\dmt)} \cite{cui:tern:osdi10,dmp:asplos09,kendo:asplos09,coredet:asplos10, dos:osdi10,grace:oopsla09}.
DMT systems ensure that the same input
is always processed with the same deterministic schedule, thus eliminating
heisenbugs and problems due to nondeterminism.  Unfortunately,
despite these efforts, an open challenge~\cite{wodet11} well recognized by
the DMT community remains: how to build \emph{both deterministic
  and efficient} DMT systems for general multithreaded programs on
commodity multiprocessors.  Existing DMT systems either incur prohibitive
overhead, or are not fully deterministic if there are data races.

Specifically, existing DMT systems enforce two forms of schedules: (1) a
\emph{mem-schedule} is a deterministic schedule of shared memory
accesses~\cite{dmp:asplos09,coredet:asplos10,dos:osdi10}, such as
\v{load/store} instructions, and (2) a \emph{sync-schedule} is a
deterministic order of synchronization
operations~\cite{kendo:asplos09,cui:tern:osdi10}, such as
\v{lock()/unlock()}.  Enforcing a mem-schedule is truly deterministic even
for programs with data races, but may incur prohibitive overhead (\eg,
roughly 1.2X-6X~\cite{coredet:asplos10}).  Enforcing a sync-schedule is
efficient (\eg, average 16\% slowdown~\cite{kendo:asplos09}) because
most code does not control synchronization and can still run in
parallel, but a sync-schedule is only deterministic for race-free
programs, when, in fact, most real programs have races, harmful or
benign~\cite{lu:concurrency-bugs,syncfinder:osdi10}.  The dilemma is,
then, to pick either determinism or efficiency but not both.

Our key insight is that although most programs have races, these races
tend to occur only within minor portions of an execution, and the majority
of the execution is still race-free.  Thus, we can resort to a
mem-schedule only for the ``racy'' portions of an execution and enforce a
sync-schedule otherwise, combining both the efficiency of sync-schedules
and the determinism of mem-schedules. We call these combined schedules
\emph{hybrid schedules}.

Based on this insight, we have built \peregrine, an efficient deterministic
multithreading system to address the aforementioned open challenge.
When a program first runs on an input, \peregrine
records a detailed execution trace
including memory accesses in case the execution runs into races.
\peregrine then \emph{relaxes} this detailed trace into a hybrid schedule,
including (1) a total order of synchronization operations and (2) a set of
execution order constraints to
deterministically resolve each occurred race.  When the same input is provided 
again, \peregrine can reuse this schedule deterministically and efficiently.

Reusing a schedule only when the program input matches exactly is too
limiting.  Fortunately, the schedules \peregrine computes are often
``coarse-grained'' and
reusable on a broad range of inputs.  Indeed, our previous work has shown
that a small number of sync-schedules can often cover over 90\% of the
workloads for real programs such as \apache~\cite{cui:tern:osdi10}.  The
higher the reuse rates, the more efficient \peregrine is.  Moreover, by reusing
schedules, \peregrine makes program behaviors more \emph{stable} across
different inputs, so that slight input changes do not lead to vastly
different schedules~\cite{cui:tern:osdi10} and thus ``\emph{input-heisenbugs}''
where slight input changes cause concurrency bugs to appear or disappear.

Before reusing a schedule on an input, \peregrine must check that the input
satisfies the \emph{preconditions} of the schedule, so that (1) the
schedule is feasible, \ie, the execution on the input will reach all
events in the same deterministic order as in the schedule and (2) the
execution will not introduce new races. (New races may occur if they are
\emph{input-dependent}; see \S\ref{sec:interthread-slice}.)  A na\"ive
approach is to collect preconditions from all input-dependent branches in
an execution trace.  For instance, if a branch instruction inspects
input variable \v{X} and goes down the true branch, we collect a
precondition that \v{X} must be nonzero.  Preconditions collected via
this approach ensures that an execution on an input satisfying the
preconditions will always follow the path of the recorded execution in all
threads.  However, many of these branches concern thread-local
computations and do not affect the program's ability to follow the
schedule. Including them in the preconditions thus unnecessarily decreases
schedule-reuse rates.

How can \peregrine compute sufficient preconditions to avoid new races and
ensure that a schedule is feasible?  How can \peregrine filter out unnecessary
branches to increase schedule-reuse rates?  Our previous
work~\cite{cui:tern:osdi10} requires developers to grovel through the
code and mark the input affecting schedules; even so, it does not
guarantee full determinism if there are data races.

\peregrine addresses these challenges with two new
program analysis techniques.  First, given an execution trace and a hybrid
schedule, it computes sufficient preconditions using
\emph{determinism-preserving slicing}, a new precondition
slicing~\cite{castro:bouncer} technique designed for multithreaded
programs.  Precondition slicing takes an execution trace and a
\emph{target} instruction in the trace, and computes 
a \emph{trace slice} that captures the instructions required for the
execution to reach the target with equivalent operand
values.  Intuitively, these instructions include ``branches whose
  outcome matters'' to reach the target and ``mutations that affect the
  outcome of those branches''~\cite{castro:bouncer}.  This trace slice
typically has much fewer branches than the original execution trace,
so that we can compute more relaxed preconditions.  However, previous
work~\cite{castro:bouncer} does not compute correct trace slices for
multithreaded programs or handle multiple targets; our slicing
technique correctly handles both cases.

Our slicing technique often needs to determine whether two pointer
variables may point to the same object.  \emph{Alias
  analysis} is the standard static technique to answer these queries.
Unfortunately, one of the best alias analyses~\cite{bddalias:pldi04} yields
overly
imprecise results for 30\% of the evaluated programs, forcing \peregrine to
reuse
schedules only when the input matches almost exactly.  The reason is that
standard
alias analysis has to be conservative and assume all possible executions,
yet \peregrine cares about alias results
according only to the executions that reuse a specific schedule.  To
improve precision, \peregrine uses \emph{schedule-guided simplification} to
first simplify a program according to a schedule, then runs standard alias
analysis on the simplified program to get more precise results.  For
instance, if the schedule dictates eight threads, \peregrine can clone
the corresponding thread function eight times, so that alias analysis can
separate the results for each thread, instead of imprecisely merging
results for all threads.

We have built a prototype of \peregrine that runs in user-space.
It automatically tracks \v{main()}
arguments, data read from files and sockets, and values
returned by \v{random()}-variants as input.  It handles long-running servers by
splitting their executions into \emph{windows} and
reusing schedules across windows~\cite{cui:tern:osdi10}.
The hybrid schedules it computes are fully deterministic for programs that
(1) have no nondeterminism sources beyond thread scheduling, data races, and
inputs tracked by \peregrine and (2) adhere to the assumptions of the tools
\peregrine uses.

We evaluated \peregrine on a diverse set of \nprog programs, including the
\apache web server~\cite{apache}; three desktop programs, such as
\pbzip~\cite{pbzip2}, a parallel compression utility; implementations of
12 computation-intensive algorithms in the popular \splash and \parsec
benchmark suites; and \racey~\cite{racy-stress}, a benchmark with numerous
intentional races for evaluating deterministic execution and replay
systems.  Our results show that \peregrine is both deterministic and efficient
(executions reusing schedules range from 68.7\% faster to 46.6\% slower
than nondeterministic executions); it can frequently reuse schedules for
half of the programs (\eg, two schedules cover all possible inputs to
\pbzip compression as long as the number of threads is the same); both its
slicing and
simplification techniques are crucial for increasing schedule-reuse rates,
and have reasonable overhead when run offline; its recording overhead
is relatively high, but can be reduced using existing
techniques~\cite{idna:vee06}; and it requires no manual efforts except a
few annotations for handling server programs and for improving precision.

Our main contributions are the schedule-relaxation approach and \peregrine, an
 efficient DMT system.  Additional contributions include
the ideas of hybrid schedules, determinism-preserving slicing, and
schedule-guided simplification.  To our knowledge, our slicing technique
is the first to compute correct (non-trivial) preconditions for
multithreaded programs.
We believe these ideas apply beyond \peregrine (\S\ref{sec:deploy}).

The remainder of this paper is organized as follows.  We first present a
detailed overview of \peregrine (\S\ref{sec:overview}).  We then describe its
core ideas: hybrid schedules (\S\ref{sec:schedule}),
determinism-preserving slicing (\S\ref{sec:slice}), and schedule-guided
simplification (\S\ref{sec:guide}).  We then present implementation issues
(\S\ref{sec:impl}) and evaluation (\S\ref{sec:eval}).  We finally discuss
related work (\S\ref{sec:related}) and conclude (\S\ref{sec:conclusion}).
