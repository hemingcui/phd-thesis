\section{Related Work} \label{sec:related}

\para{Deterministic execution.}
By reusing schedules, \peregrine mitigates input nondeterminism and makes
program behaviors repeatable across inputs.  This method is based on the
\emph{schedule-memoization} idea in our previous work
\tern~\cite{cui:tern:osdi10}, but \peregrine largely eliminates manual annotations,
and provides stronger determinism guarantees than \tern.  To our
knowledge, no other DMT systems mitigate input nondeterminism; some
actually aggravate it, potentially creating ``input-heisenbugs.''

\peregrine and other DMT systems can be complementary: \peregrine can use an
existing DMT algorithm when it runs a program on a new input
so that it may compute the same schedules at different sites;
existing DMT systems can speed up their pathological cases using the
schedule-relaxation idea.

Determinator~\cite{determinator:osdi10}
advocates a new, radical programming model that converts all races,
including races on memory and other shared resources, into exceptions, to
achieve pervasive determinism.  This programming model is not designed to
be backward-compatible.  dOS~\cite{dos:osdi10} provides similar pervasive
determinism with backward compatibility, using a \dmt algorithm first
proposed in~\cite{dmp:asplos09} to enforce mem-schedules.  While \peregrine
currently focuses on multithreaded programs, the ideas in \peregrine can be
applied to other shared resources to provide pervasive determinism.
\peregrine's hybrid schedule idea may help
reduce dOS's overhead.  Grace~\cite{grace:oopsla09} makes multithreaded
programs with fork-join parallelism behave like sequential programs.  It
detects memory access conflicts efficiently using hardware page
protection.  Unlike Grace, \peregrine aims to make general multithreaded
programs, not just fork-join programs, repeatable.  

Concurrent to our work, \dthreads~\cite{dthreads:sosp11} is another
efficient DMT system.  It tracks memory modifications using hardware page
protection and provides a protocol to deterministically commit these
modifications.  In contrast to \dthreads, \peregrine is software-only and does not
rely on page protection hardware which may be expensive and suffer from
false sharing; \peregrine records and reuses schedules, thus it can handle
programs with ad hoc synchronizations~\cite{syncfinder:osdi10} and make
program behaviors stable.

\para{Program analysis.}  Program slicing~\cite{Tip:slicing} is a general
technique to prune irrelevant statements from a program or trace.
Recently, systems researchers have leveraged or invented slicing
techniques to block malicious input~\cite{castro:bouncer}, synthesize
executions for better error diagnosis~\cite{esd:eurosys10}, infer source
code paths from log messages for postmortem
analysis~\cite{sherlog:asplos10}, and identify critical inter-thread reads
that may lead to concurrency errors~\cite{conseq:asplos11}.  
Our determinism-preserving slicing technique produces a correct trace slice for
multithreaded programs and supports multiple ordered targets.  It thus has
the potential to benefit existing systems that use slicing.  

Our schedule-guided simplification technique shares similarity with
SherLog~\cite{sherlog:asplos10} such as the removal of branches
contradicting a schedule.  However, SherLog starts from log messages and
tries to compute an execution trace, whereas \peregrine starts with a schedule
and an execution trace and computes a simplified yet runnable program.
\peregrine can thus transparently improve the precision of many existing
analyses: simply run them on the simplified program.

%% Moreover, our scheduled-guided analysis helps compute
%% accurate alias results by simplifying a program based on a sequence of
%% dynamic events, thus it may improve the precision of systems that also
%% have this need, such as Bouncer~\cite{castro:bouncer} and
%% SherLog~\cite{sherlog:asplos10}.
% In addition, the simplified programs our technique computes can be used
% for other analysis.


%%   some of existing slicing also uses alias, but rather generic alias.  our
%%   schedule guided alias will improve their precision

%%   sherlog: tied to analysis.  they didn't use alias or use off the shell
%%   alias.
  
%%   log messages: more sparse than sync.  we start with detailed trace.

%%   we generate a program so that can run arbitrary analysis.

%%   enhancer may help.


\para{Replay and re-execution.}  Deterministic
replay~\cite{r2:osdi,friday2007,srinivasan:flashback,revirt,dejavu,vmware-record-replay,smp-revirt:vee08,pres:sosp09,scribe:sigmetrics10,odr:sosp09,capo:asplos09}
aims to replay the exact recorded executions, whereas \peregrine ``replays''
schedules on different inputs.  Some recent deterministic replay systems
include Scribe, which tracks page ownership to enforce deterministic
memory access~\cite{scribe:sigmetrics10}; Capo, which defines a novel
software-hardware interface and a set of abstractions for efficient
replay~\cite{capo:asplos09}; PRES and ODR, which systematically search for
a complete execution based on a partial one~\cite{pres:sosp09,odr:sosp09};
SMP-ReVirt, which uses page protection for recording the
order of conflicting memory accesses~\cite{smp-revirt:vee08}; and
Respec~\cite{respec:asplos10}, which uses online replay to keep multiple
replicas of a multithreaded program in sync.  Several
systems~\cite{pres:sosp09,respec:asplos10} share the same insight as \peregrine:
although many programs have races, these races tend to occur infrequently.

\peregrine can help these systems reduce CPU, disk, or network bandwidth
overhead, because for inputs that hit \peregrine's schedule cache, these systems
do not have to record a schedule.

Retro~\cite{retro:osdi10} shares some similarity with \peregrine because it also
supports ``mutated'' replay.  When repairing a compromised system, Retro
can replay legal actions while removing malicious ones using a novel
dependency graph and \emph{predicates} to detect when changes to an object
need not be propagated further.  \peregrine's determinism-preserving slicing
algorithm may be used to automatically compute these predicates, so that
Retro does not have to rely on programmer annotations. 
% Similarly, \peregrine may leverage Retro's system architecture to implement
% pervasive determinism across the OS.

\para{Concurrency errors.} The complexity in developing multithreaded
programs has led to many concurrency errors~\cite{lu:concurrency-bugs}.
Much work exists on concurrency error
detection, diagnosis, and correction (\eg,~\cite{yu:racetrack:sosp,racerx:sosp03,lu:muvi:sosp,conmem:asplos10,conseq:asplos11,2ndstrike:asplos11,linearizable:eurosys11,ctrigger:asplos09}).  \peregrine aims to make the
%% detection~\cite{yu:racetrack:sosp,savage:eraser,racerx:sosp03,lu:muvi:sosp,avio:asplos06,conmem:asplos10,conseq:asplos11,2ndstrike:asplos11,linearizable:eurosys11},
%% diagnosis~\cite{racefuzzer:pldi08,ctrigger:asplos09,atomfuzzer:fse08}, and
%% correction~\cite{dimmunix:osdi08,gadara:osdi08}.  \peregrine aims to make the
executions of multithreaded programs repeatable, and is complementary to
existing work on concurrency errors.  \peregrine may use existing
work to detect and fix the errors in the schedules it computes.
Even for programs free of concurrency errors, \peregrine still provides value by
making their behaviors repeatable.



