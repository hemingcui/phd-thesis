\section{Schedule-Guided Simplification} \label{sec:peregrine-guide}

In both the inter- and intra-thread steps of determinism-preserving
slicing, \peregrine frequently queries alias analysis.  The inter-thread step
needs alias information to determine whether two instructions may race
(\vv{mayalias()} in Figure~\ref{fig:peregrine-detect-input-race}).  The intra-thread step
needs alias information to track potential dependencies.

We thus integrated \bddbddb~\cite{bddbddb,bddalias:pldi04}, one of the best alias analyses,
into \peregrine by creating an LLVM frontend to collect program facts into
the format \bddbddb expects.  However, our initial evaluation showed that
\bddbddb sometimes yielded overly imprecise results, causing \peregrine to prune
few branches, reducing schedule-reuse rates (\S\ref{sec:peregrine-stable}).  The
cause of the imprecision is that
standard alias analysis is purely static, and has to be conservative and
assume all possible executions.  However, \peregrine requires alias results
only for the executions that may reuse a schedule, thus suffering
from unnecessary imprecision of standard alias analysis.


To illustrate, consider the example in Figure~\ref{fig:peregrine-example}.  Since the
number of threads is determined at runtime, static analysis has to abstract
this unknown number of dynamic thread instances, often coalescing
results for multiple threads into one.  When \peregrine slices the trace in
Figure~\ref{fig:peregrine-trace}, \bddbddb reports that the accesses to \vv{data} 
(L13 instances) in different threads may alias.
\peregrine thus has to add them to the trace slice to avoid new races
(\S\ref{sec:peregrine-interthread-slice}).  Since L13 depends 
on L12, L11, and L10, \peregrine has to add them to the trace slice, too.
Eventually, an imprecise alias result snowballs into a slice
as large as the trace itself.  The preconditions from this slice
constrains the data size to be exactly 2, so \peregrine cannot reuse
the hybrid schedule in Figure~\ref{fig:peregrine-trace} on other data sizes.


To improve precision, \peregrine uses schedule-guided simplification to simplify
a program according to a schedule, so that 
alias analysis is less likely to get confused.  
Specifically, \peregrine performs three main simplifications:

\begin{enumerate}

\item It clones the functions as needed.  For instance, it gives
  each thread in a schedule a copy of the thread function.

\item It unrolls a loop when it can determine the loop bound based on a
  schedule.  For instance, from the number of the \vv{pthread\_create()}
  operations in a schedule, it can determine how many times the loop at
  lines L4--L5 in Figure~\ref{fig:peregrine-example} executes.

\item It removes branches that contradict the schedule.  Loop unrolling
  can be viewed as a special case of this simplification.

\end{enumerate}

\peregrine does all three simplifications using one algorithm.  From a high
level, this algorithm iterates through the events in a schedule.  For each
pair of adjacent events, it checks whether they are ``at the same level,''
\ie, within the same function and loop iteration.  If so, \peregrine does
not clone anything; otherwise, \peregrine clones the mismatched portion of
instructions between the events.  (To find these instructions, \peregrine uses an
interprocedural reachability analysis by traversing the control flow graph
of the program.)  Once these simplifications are applied, \peregrine can further
simplify the program by running stock LLVM transformations such as
constant folding.  It then feeds the simplified program to \bddbddb,
which can now
distinguish different thread instances (\emph{thread-sensitivity} in
programing language terms) and precisely reports that L13
of $t_0$ and L13 of $t_1$
are not aliases, enabling \peregrine to compute the small trace slice in
Figure~\ref{fig:peregrine-trace}.

By simplifying a program, \peregrine can automatically improve the precision of
not only alias analysis, but also other analyses.  We have
implemented \emph{range analysis}~\cite{range:pldi00} to improve
the precision of alias analysis on programs
that divide a global array into disjoint partitions, then process each
partition within a thread.
The accesses to these disjoint partitions from different threads do
not alias, but \bddbddb often collapses the elements of an array into one
or two abstract locations, and reports the accesses as aliases.  Range
analysis can solve this problem by tracking the lower and upper bounds of
the integers and pointers.  With range analysis, \peregrine answers alias queries
as follows.  Given two pointers
\vv{(p+i)} and \vv{(q+i)}, it first queries \bddbddb whether \vv{p} and \vv{q} may alias.
If so, it queries the more expensive range analysis whether \vv{p+i} and
\vv{q+j} may be equal.  It considers the pointers as aliases only when both
queries are true.  Note that our simplification technique is again key
to precision because standard range analysis would merge ranges of different threads
into one.


While schedule-guided simplification improves precision, \peregrine has to run
alias analysis for each schedule, instead of once for the program.  This
analysis time is reasonable as \peregrine's analyzer runs offline.  Nonetheless,
the simplified programs \peregrine computes for different schedules are largely
the same, so a potential optimization is to \emph{incrementally} analyze a
program, which we leave for future work.

