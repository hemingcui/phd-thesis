\section{Implementation Issues} \label{sec:peregrine-impl}

This section discusses implementation issues not covered by previous sections.

\subsection{Recording an Execution} \label{sec:peregrine-record}

To record an execution trace, \peregrine can use one of the existing
deterministic record-replay
systems~\cite{scribe:sigmetrics10,smp-revirt:vee08,idna:vee06} provided
that \peregrine can extract an instruction trace.  For simplicity, we have built
a crude recorder on top of the LLVM interpreter in \klee.  When an program
calls the \peregrine-provided wrapper to \vv{pthread\_create(..., func, args)},
the recorder spawns a thread to run \vv{func(args)}
within an interpreter instance.  These interpreter instances log
each instruction interpreted into a central file.
For simplicity, \peregrine does symbolic execution during recording because it
already runs \klee when recording an execution and pays the high overhead
of interpretation.  A faster recorder would enable \peregrine to
symbolically execute only the trace slices instead of the typically larger
execution traces.  Since deterministic record-replay is a well
studied topic, we have not focused our efforts on optimizing the recorder.


\subsection{Handling Blocking System Calls}

Blocking system calls are natural scheduling points, so \peregrine includes them
in the schedules~\cite{cui:tern:osdi10}.  It currently considers eight
blocking system calls, such as \vv{sleep()},
\vv{accept()}, and \vv{read()}.  For each blocking system call, the recorder
logs when the call is issued and when the call is returned.  When \peregrine
computes a schedule, it includes these blocking system call and return
operations.  When reusing a schedule, \peregrine attempts to enforce the same
call and return order.  This method works well for blocking system calls
that access local state, such as \vv{sleep()} or \vv{read()} on local file
descriptors.  However, other blocking system calls receive input from
the external world, which may or may not arrive each time a
schedule is reused.  Fortunately, programs that use these operations tend to be
server programs, and \peregrine handles this class of programs differently.

\subsection{Handling Server Programs} \label{sec:peregrine-window}

Server programs present two challenges for \peregrine.  First, they are more
prone to timing nondeterminism than batch programs because their inputs
(client requests) arrive nondeterministically.  Second, they often run
continuously, making their schedules too specific to reuse.  

\peregrine addresses these challenges with the \emph{windowing} idea from our
previous work~\cite{cui:tern:osdi10}.  The insight is that server programs
tend to return to the same quiescent states.  Thus, instead of processing
requests as they arrive, \peregrine\ breaks a continuous request stream down into
windows of requests.  Within each window, it admits requests only at fixed
points in the current schedule.  If no requests arrive at an admission
point for a predefined timeout, \peregrine\ simply proceeds with the partial
window.  While a window is running, \peregrine\ buffers newly arrived requests
so that they do not interfere with the running window.  With windowing,
\peregrine can record and reuse schedules across windows.

\peregrine requires developers to annotate points at which request processing begins and
ends.  It also assumes that after a server processes all current requests,
it returns to the same quiescent state.  That is, the input from
the requests does not propagate further after the requests are processed.
The same assumption applies to the data read from local files.
For server programs not meeting this assumption, developers can manually
annotate the functions that observe the changed server state, so that \peregrine
can consider the return values of these functions as input.
For instance, since \apache caches client requests, we made it work with \peregrine
by annotating the return of \vv{cache\_find()} as input.

One limitation of applying our \peregrine prototype to server programs is that our current
implementation of schedule-guided simplification does not work well with
thread pooling.  To give each thread a copy of the corresponding thread
function, \peregrine identifies \vv{pthread\_create(...,func,...)} operations in
a program and clones function \vv{func}.  Server programs that use
thread pooling tend to create worker threads to run generic thread
functions during program initialization, then repeatedly use the threads
to process client requests.  Cloning these generic thread functions thus
helps little with precision.  One method to solve this problem is to clone
the relevant functions for processing client requests.  We have not
implemented this method because the programs we evaluated include only one server
program, \apache, on which slicing already performs reasonably well
without simplification (\S\ref{sec:peregrine-stable}).


\subsection{Skipping Wait Operations} \label{sec:peregrine-nowait}

When reusing a schedule, \peregrine enforces a total order of synchronization
operations, which subsumes the execution order enforced by the original
synchronization operations.  Thus, for speed, \peregrine can actually skip the
original synchronization operations as in~~\cite{cui:tern:osdi10}.
\peregrine currently skips sleep-related operations
such as \vv{sleep()} and wait-related operations such as
\vv{pthread\_barrier\_wait()}.  These operations often unconditionally block
the calling thread, incurring context switch overhead, yet this blocking
is unnecessary as \peregrine already enforces a correct execution order.  Our
evaluation shows that skipping blocking operations significantly speeds
up executions.

\subsection{Manual Annotations} \label{sec:peregrine-func-summary}

\peregrine works automatically for most of the programs we evaluated.  However,
as discussed in \S\ref{sec:peregrine-window}, it requires manual annotations for
server programs.  In addition, if a program has nondeterminism sources
beyond what \peregrine automatically tracks, developers should annotate these
sources with \vv{input(void* addr, size\_t nbyte)} to mark \vv{nbyte} of
data starting from \vv{addr} as input, so that \peregrine can track this
data.

Developers can also supply optional annotations to improve \peregrine's
precision in four ways.  First, for better alias results, developers can
add custom memory allocators and \vv{memcpy}-like functions to a
configuration file of \peregrine.  Second, they can help \peregrine better track
ranges by adding \vv{assert()} statements.  For instance, a function in the
FFT implementation we evaluated uses bit-flip operations to transform an
array index into another, yet both indexes have the same range.  The range
analysis we implemented cannot precisely track these bit-flip operations,
so it assumes the resultant index is unbounded.  Developers can fix this
problem by annotating the range of the index with an assertion
``\vv{assert(index<bound)}.''  Third, they can provide \emph{symbolic
  summaries} to help \peregrine compute more relaxed constraints.  For instance,
consider Figure~\ref{fig:peregrine-precond} and a typical implementation of
\vv{atoi()} that iterates through all characters in the input string and
checks whether each character is a digit.  Without a summary of \vv{atoi()},
\peregrine would symbolically execute the body of \vv{atoi()}.  The preconditions
it computes for \vv{argv[3]} would be $ (argv_{3,0} \neq 49) \wedge
(argv_{3,1} < 48 \vee argv_{3,1} > 57)$, where $argv_{3,i}$ is the $i$th
byte of \vv{argv[3]} and 48, 49, and 57 are ASCII codes of `0', `1', and
`9'.  These preconditions thus unnecessarily constrain \vv{argv[3]} to
have a valid length of one.  Another example is string search.  When a
program calls \vv{strstr()}, it often concerns whether there exists a
match, not specifically where the match occurs.  Without a symbolic
summary of \vv{strstr()}, the preconditions from \vv{strstr()} would
constrain the exact location where the match occurs.  Similarly, if a
trace slice contains complex code such as a decryption function, users can
provide a summary of this function to mark the decrypted data as symbolic
when the argument is symbolic.  Note that complex code not included in
trace slices, such as the \vv{read()} in Figure~\ref{fig:peregrine-example}, is not an
issue.

