\subsubsection{State Space Exploration}
\dbug~\cite{dbug:spin11} is a tool for systematic testing of concurrent
Linux programs. The goal of \dbug is to explore the state space of
different program states of a concurrent program by systematically
enumerating different orders in which concurrent POSIX threads and
Linux sockets calls can be exercised by a program test.

To this end, \dbug abstractly represents the state space of a program
test using an \emph{execution tree}. Nodes of the execution tree
represent non-deterministic choice points and edges record
non-deterministic choices representing program state transitions. A
branch leading from the root of the tree to a leaf thus encodes a
unique test execution as a sequence of non-deterministic scheduling
choices.

\dbug uses this abstraction to repeatedly execute a program test,
recording what sequence of program state transition have been explored
by each iteration and guaranteeing that different iterations explore
different sequences.

\subsubsection{State Space Reduction}
To mitigate the combinatorial explosion resulting from having to
consider all possible permutations of concurrently enabled
transitions, \dbug implements dynamic partial order
reduction (DPOR)~\cite{flanagan:dynamicpo}.

A key building block of DPOR is the \emph{independence} relation that
identifies pairs of commutitative program state transitions. To avoid
the prohibitive overhead of having to track each memory access to
compute the independence relation precisely, \dbug adopts the approach
used by RacePro~\cite{racepro:sosp11}. In particular, for each program state
transition, \dbug records what abstract resources this transition
accesses, and uses this information to approximate the independence
relation.

\subsubsection{State Space Estimation}
To estimate the progress of the exploration, \dbug uses the
\emph{weighted backtrack estimator} (WBE)~\cite{Kilby2006} to estimate
the size of a partially explored execution tree. WBE is an online
variant of Knuth's offline technique for tree size
estimation~\cite{Knuth1975}. WBE uses the length of each explored branch
weighted by the probability it is explored (assuming uniform
distribution over edges) to predict the size of the tree. Formally,
WBE updates the estimate every time a branch of the execution is
explored, setting the estimate to:
\[
\mathit{estimate} =
\frac{\sum\limits_{b \in B} p(b) * p^{-1}(b)}{\sum\limits_{b \in B} p(b)} =
\frac{|B|}{\sum\limits_{b \in B} p(b)}
\]
where $B$ is the set of explored branches, and $p(b)$ is the
probability of exploring the branch. For a branch $b = (v_1, \ldots,                                                                                          
v_n)$, where $v_i$ is the $i$-th node along the branch $b$, the
probability $p(b)$ is defined as: \[\prod\limits_{i = 1}^{n - 1}
\frac{1}{E(v_i)}\] where $E(v_i)$ is the number of enabled
edges originating from node $v_i$.

\subsubsection{Source Code Annotations}
In general, \dbug does not require any changes to the source code. In
fact, as \dbug uses run-time interposition to control the order in
which concurrent POSIX threads and Linux sockets calls happen, it does
not require access to the source code at all.  However, similar to the
SysNam annotations, in some cases, source code annotations can improve
the efficiency of \dbug.

For example, a programmer can use the {\tt \dbug\_off()} and {\tt
  \dbug\_on()} annotations to exclude parts of the tested program from
the scope of \dbug, avoiding exploration of uninteresting call
interactions. (Example: using a lock to increment a variable in
pfscan). Further, a programmer can use the {\tt int \dbug\_nondet(int
  event)} function to identify a non-deterministic event. The first
time such an event occurs, \dbug records its value and makes sure the
same value is used for the same event in future iterations. (Example:
Recording the return values of {\tt gettimeofday()} for redis).
Finally, a programmer can use the {\tt int \dbug\_choose(int n)}
function to non-determinstically choose a value between $0$ and
$n-1$. (Example: Simulating timeouts or system call failures).
