\section{Handling Diverse Programming Constructs}

\subsection{Network Operations}

list of network operations we handle:

\subsection{Timeouts} \label{sec:timeout}

Real-world programs frequently use timeouts (\eg, \v{sleep}, \v{epoll},
and \v{pthread\_cond\_timedwait}) for periodic activities or timed waits.
Yet most previous systems~\cite{xxx} ignored timeouts, causing executions to
be nondeterministic because timeouts rely on real time.  Worse, they
easily cause deadlocks.  One example deadlock in our experiments was
running \pbzip with previous system \dthreads.  \pbzip deadlocked in
\v{pthread\_cond\_timedwait} because \dthreads treats
\v{pthread\_cond\_timedwait} just as \v{pthread\_cond\_wait}, but \pbzip
sometimes relies on the timeout to terminate.

\parrot makes timeouts deterministic by proportionally converting them to a
logical.  When a thread registers a relative timeout that fires $\Delta
t_p$ later in real time, \parrot converts $\Delta t_p$ to a relative logical
timeout $\Delta t_p /R$ where $R$ is a configurable conversion ratio
default to 3 $\mu$s, which works for all evaluated programs that use
timeouts.

When all threads in the current process are on the wait queue, \parrot spawns
an idle thread to keep the logical time going. The thread simply
repeatedly gets the turn, sleeps for time $R$, and gives up the turn.  An
alternative is fast-forwarding~\cite{modist:nsdi09}.  Our experiments show
that using an idle thread has much better performance than fast-forwarding
because fast-forwarding frequently wakes up threads prematurely when the
external events the threads are waiting for have not arrived yet, wasting
CPU cycles.

Proportional conversion is better than a fixed logical timeout because it
matches developer intents better.  For instance, developers may want to
performance one important periodical activity with higher frequency, and a
less important one with lower frequency.  Proportional conversion keeps
the relative ratio of the frequencies, and improves performance based on
our experiments.  Of course, \parrot's logical time corresponds loosely to
real time, and may be less useful for real-time applications.
dOS~\cite{dos:osdi10} discussed the possibility of converting real to
logical time, but it did not present a formula.  Moreover, it
fast-forwards logical time after all threads block, slower than using an
idle thread.

\pthread timed synchronizations such as \v{pthread\_cond\_timedwait} use
absolute real time, so \parrot cannot compute $\Delta t_p$.  To solve this
problem, developers can call a \parrot-provided function \v{set\_base\_time}
to inform \parrot the base time, and \parrot uses the difference between the
base time and the time passed to \v{pthread\_cond\_timedwait} as $\Delta
t_p$.

list of timeouts \parrot handles

because they depend on the inherently nondeterministic physical time.




not only nondeterministic executions, but also
Programs frequently use these operations for periodic activities or timed
waits.

simply be that they did not appear in the evaluation benchmarks of the systems


  physical time in environment
    kernel
    network lease
  performance, sleep fsync in redis

  Q: why can't we make f(t) = C?

