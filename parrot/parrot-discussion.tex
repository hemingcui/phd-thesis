\section{Determinism Discussion} \label{sec:parrot-discussion}

\parrot's determinism is relative to three factors: (1) external input (data
and timing), (2) \nondets, and (3) data races \wrt
the enforced synchronization schedules.  Factor 1 is inherently
nondeterministic, and \parrot mitigates it by reusing schedules on inputs.
Factor 2 is developer-intended.  Factor 3 can be easily eliminated, but we
deemed it not worthwhile.  Below we explain how to make data races
deterministic in \parrot and why it is not worthwhile.

We designed a simple memory commit protocol to make data races
deterministic in \parrot, similar to those in previous
work~\cite{grace:oopsla09,determinator:osdi10,dthreads:sosp11}.  Each
thread maintains a private, copy-on-write mapping of shared memory.  When
a thread has the turn, it commits updates and fetches other threads'
updates by merging its private mapping with shared memory.  Since only one
thread has the turn, all commits are serialized, making data races
deterministic.  (Threads running nondeterministically in performance
critical sections access shared memory directly as intended.)
This protocol may also improve speed
by reducing false sharing~\cite{dthreads:sosp11}.  Implementing it can
leverage existing code~\cite{dthreads:sosp11}.

%% We deemed the effort not worthwhile for three reasons.  
%% First, prior work has shown that data races rarely occur if a
%% synchronization schedule is enforced.  For instance,
%% \peregrine~\cite{peregrine:sosp11} reported that at most 10 races in
%% millions of shared memory accesses within an execution.  Second, the
%% races that occured are often ad hoc synchronizations (\eg,
%% ``\v{while(flag)};'')~\cite{syncfinder:osdi10}.  To handle these benign races,
%% prior systems often require manual annotations
%% anyway~\cite{dthreads:sosp11, grace:oopsla09}.  Third, the remaining races
%% are so rare that we may be able to afford the nondeterminism they introduce.  For
%% instance, to reproduce failures caused by them, we can search through a
%% small number of executions (\eg, $<$ 96 to reproduce an \apache race
%% caused by a real workload~\cite{pres:sosp09}).  Similarly, the races can be detected
%% by checking a small number of executions using model checkers~\cite{musuvathi:chess:osdi08}. In short, as long as the number of schedules for
%% an input is small, the problems caused by nondeterminism are easy to
%% solve.

We deemed the effort not worthwhile for three reasons. First, making data
races deterministic is often costly.  Second, many races are \emph{ad hoc
  synchronizations} (\eg, ``\v{while(flag)};'')~\cite{syncfinder:osdi10}
which require manual annotations anyway in some previous systems that make
races deterministic~\cite{dthreads:sosp11, grace:oopsla09}.  Third, most
importantly, we believe that stability is much more useful for reliability
than full determinism: once the set of schedules is much reduced, we can
afford the nondeterminism introduced by a few data races.  Specifically,
previous work has shown that data races rarely occur if a synchronization
schedule is enforced.  For instance, \peregrine~\cite{peregrine:sosp11}
reported at most 10 races in millions of shared memory accesses
within an execution.  To reproduce failures caused by the few races, we
can search through a small set of schedules (\eg, fewer than 96 for
an \apache race caused by a real workload~\cite{pres:sosp09}).  Similarly,
we can detect the races by model checking a small set of
schedules~\cite{musuvathi:chess:osdi08}. In short, by vastly reducing
schedules, \smt makes the problems caused by nondeterminism easy to solve.

%% Making data races deterministic in
%% \parrot is straightforward.  We have designed a memory commit protocol to do
%% so, similar to the protocols in \grace, \determinator, and \dthreads.  In
%% our protocol, each thread maintains a private copy of the shared memory,
%% and \parrot mains a common copy.  Copy-on-write can be used to reduce memory
%% and CPU
%% overhead~\cite{grace:oopsla09,determinator:osdi10,dthreads:sosp11}.  When
%% a thread has the turn, it synchronizes its copy of shared memory with the
%% common copy by merging the updates, as shown in Figure~\ref{fig:commit}.
%% Threads executing outside nondeterministic regions commit in serial
%% because at most one thread may have the turn.  Threads executing in
%% nondeterministic regions access the common copy directly.  Such a protocol
%% sometimes improves speed by reducing false sharing~\cite{dthreads:sosp11}.

%% Implementing this protocol is easy.  Since it is very similar to existing
%% protocols, we may even leverage \dthreads's code.  However, we decided
%% that it is not worthwhile to implement this protocol.  The reasons are
%% threefold.  First, prior work has shown that data races rarely occur if a
%% synchronization schedule is enforced.  For instance
%% PRES~\cite{pres:sosp09} reported that
%% XXX. \peregrine~\cite{peregrine:sosp11} reported that XXX.  Second, the
%% occurred races are often ad hoc synchronizations~\cite{syncfinder:osdi10}.
%% These benign races cannot be handled by some prior systems designed to
%% make data races deterministic~\cite{xxx}, and need manual annotations
%% anyway.  Third, the remaining races are so rare that we can afford the
%% nondeterminism they create.  For instance, to reproduce failures caused by
%% them, we can afford to search through a few executions.  PRES reported
%% XXX.  Similarly, we can detect these races by running model checkers such
%% as CHESS and \dbug.  In short, as long as the number of schedules for an
%% input is small, the problems caused by nondeterminism are easy to solve.

%% Due to these reasons, the complexity of implementing a memory commit
%% protocol is not worthwhile.  Such a protocol often has to make unrealistic
%% assumptions, severely limiting applicability.  For instance, our
%% experiments show that \dthreads failed to support XXX programs because of
%% the assumptions of its memory commit protocol.







%% because races occur rarely
%% when synchronization schedules are
%% enforced~\cite{peregrine:sosp11,pres:sosp09}, and can be
%% detected~\cite{musuvathi:chess:osdi08} or
%% reproduced~\cite{odr:sosp09,pres:sosp09} using better approaches

%% and as long as the number of schedules for an input is small, the problems
%% caused by nondeterminism are easy to solve.



