\section{Performance Hint Abstractions} \label{sec:hints}

\parrot provides two performance-hint abstractions: a \emph{soft barrier} and
a \emph{performance critical section}.  This section describes these abstractions and
their usage.

%% These simple hints effectively
%% improved \parrot's performance (\S\ref{sec:eval}), and have the potential to
%% benefit other DMT or \smt systems and even classic nondeterministic
%% schedulers.  

%% \compute and \nondet. \compute marks the starting point of a thread's
%% computation participating in a parallel computation group.  It lets \parrot
%% switch to an efficient schedule that runs all computations in the group in
%% parallel.  It does not affect determinism.  \nondet marks a code region
%% upon which \parrot should revert back to nondeterministic execution.  This
%% hint may trade off a small amount of determinism for speed, so it should
%% be applied with caution, only when (1) the code region, if run
%% deterministically, has high overhead and (2) the additional schedules have
%% been thoroughly checked (\eg, by tools or developers inspecting the code
%% region).  Mixing deterministic and nondeterministic executions is tricky
%% because they may interfere; we have carefully defined the semantics of our
%% performance hints to provably avoid interference (\S\ref{sec:hints}).


\subsection{Soft Barrier} \label{sec:soft-barrier}

A \vcompute encourages the scheduler to coschedule a group of threads at
given program points.  It is for performance only, and a scheduler can
ignore it without affecting correctness.  It operates as a
barrier with deterministic timeouts in \parrot, helping \parrot switch to faster
schedules that avoid serializing parallel computations.  The interface is
\vspace{-3 mm}
\lgrindfile{parrot/code/compute.cpp}
\vspace{-1 mm}

%% The \v{soba\_init} function initializes a soft barrier.  \v{group\_size}
%% specifies the number of threads in the coscheduling group.  \v{key} is 
%% an opaque key identifying a coscheduling group.
%% An absent key refers to a unique anonymous barrier.
%% \v{timeout} is an optional relative timeout overriding the default timeout.  \parrot
%% ensures that the timeout is deterministic (\S\ref{sec:scheduler}).
%% The \v{soba\_wait} function waits for all \v{group\_size} threads to arrive
%% until some waiting thread times out.  If a thread times out,
%% it releases all threads currently waiting for the barrier to
%% get partial coscheduling benefits~\cite{partial-barrier:atc06}. (We need \v{soba\_init},
%% because it may be difficult to figure out the group size when \v{soba\_wait} is called.)

\noindent
One thread calls \v{soba\_init(N, key, timeout)} to initialize the barrier
named \v{key}, logically indicating that a group of \v{N} threads will
be spawned.  Subsequently, any thread which calls \v{soba\_wait(key)}
will block until either (1) \v{N}-1 other threads have also called
\v{soba\_wait(key)} or (2) \v{timeout} time has elapsed since the first
thread arrived at the barrier.  This timeout is made deterministic by \parrot
(\S\ref{sec:scheduler}).  \v{soba\_init} can be called multiple
times: if the number of coscheduled threads varies but is known at runtime,
the soft barrier can be initialized before each use.  Both \v{key} and
\v{timeout} in \v{soba\_init} are optional.  An absent \v{key} refers to a
unique anonymous barrier.  An absent \v{timeout} initializes the barrier
with the default timeout.

A \compute may help developers express coscheduling intent to classic
nondeterministic schedulers~\cite{coschedule}.  One advantage is that it
makes the group of threads and program points explicit.  It is more robust
to developer mistakes than a real barrier~\cite{coschedule:sigmetrics96}
for coscheduling purposes because schedulers cannot ignore a real barrier.

%% \begin{tightenum}

%% \item[$\bullet$] Data partition. Data is partitioned among worker
%%   threads, and each worker thread computes on a partition. This pattern is
%%   the most common; XXX out of the \nprog programs follow this pattern,
%%   including the programs with fork-join parallelism.  Most programs with
%%   this pattern need no soft barriers because the default schedules already
%%   align the computations.  In rare cases (\eg, when another thread
%%   interferes with the worker threads like in Figure~\ref{fig:example} or
%%   the worker threads do synchronizations asymmetrically before starting
%%   the computations), developers can initialize a soft barrier to the
%%   number of worker threads, and add \v{soba\_wait} before each worker's
%%   computation.  These soft barriers often work extremely well because
%%   developers strive to partition data evenly and keep all threads busy.

%% \item[$\bullet$] Pipeline. The threads are split into stages of a
%%   pipeline, and each item in the workload flows through the pipeline
%%   stages.  \ferret, \dedup, \vips, and \xtwosixfour follow this pattern.
%%   These programs often need soft barriers for good performance because
%%   threads have different roles and thus do different synchronizations,
%%   causing default round-robin scheduling to serialize computations.  The
%%   methodology to add soft barriers is to align the most time-consuming
%%   computations during the steady stage of the pipeline.  For instance, for
%%   \vips and \xtwosixfour, we aligned all their stages.  For \ferret, we
%%   aligned the two most time-consuming stages (\v{t\_vec} and \v{t\_rank})
%%   out of six.  \dedup needs no barriers.  How well soft barriers work
%%   depends on (1) load imbalance of different phases and (2) soft barriers
%%   timeouts during pipeline startup and teardown.

%% \item Map reduce. Programs with this pattern use both data partition
%%   and pipeline, so the methodology follows both.  In general, the mapper
%%   threads should be aligned.

%%   The reducer threads should be aligned with   the mapper if the reducers 

%%   Depending on how fast the mapper and the reducer threads are, the
%%   reducer threads

%%   combines

%%   All the seven mapreduce programs in \phoenix fall into
%%   this category.  For all the sever but one (\wordcount), we only need to
%%   add compute hints to the beginning of each \v{map} function for all
%%   worker threads, because the compuation amount of each \v{reduce}
%%   function is much smaller than the \v{map} function. For \wordcount, we
%%   need \compute hints for both the two functions.

%% \item Workpile.  The workload consists of a pile of independent work
%%   items, processed by worker threads running in parallel.  Among the
%%   programs we evaluated, \bdb, \openldap, \redis, \pfscan, and \aget fall
%%   in this category.  

%%   All but one (\pfscan, needs \nondet hints) in this
%%   category have been working well with our default scheduler.

%% \end{tightenum}

%% Computation hints encourage the runtime to align the execution of a thread
%% group to be lockstep parallel at given points.  They are operationally
%% barriers with deterministic timeouts. The interface is
%% % They may benefit even nondeterministic thread runtimes


%% \lgrindfile{code/compute.cpp}

%% \noindent
%% Function \v{compute\_init} initializes information about the thread group.
%% \v{group\_size} specifies the number of threads.  \v{key} is an optional
%% opaque key identifying the ``barrier'' in case the program needs multiple
%% such barriers.  An absent key refers to a unique anonymous barrier.
%% \v{timeout} is an optional deterministic timeout overriding the maximum
%% amount of time the runtime waits for \v{group\_size} threads to arrive.
%% We needed to set this timeout for only one out of all evaluated programs.
%% Function \v{compute} is similar to a barrier wait except it can time out.
%% \v{compute} could have an additional argument specifying the estimated
%% size of the computation for more fine-grained alignment, but we did not
%% find the need for it to get reasonable performance
%% (\S\ref{sec:performance}).

%% Computation hints switch schedules without introducing nondeterminism.
%% They can be implemented in other DMT or \smt systems to improve speed.
%% Moreover, they can benefit nondeterministic thread runtimes as
%% well~\cite{xxx}.

%% usage:
%%   general usage: wait time or vtunes
%%   parallel programming patterns
%%     data parallel: can have load balance as well
%%     pipeline: steady stage, start up \& tear down stage

%% experience: 
%%   simple, effective
%%   doesn't need much time to add, low overhead already
%%     developers can do better job: both faster in time spent in adding hints, and lower overhead at runtime).
%%   doesn't not need to be 100\%
%%   doesn't need to solve all serialization

%% \compute helps \parrot switch to a faster schedule that makes intended
%% parallel computations run in parallel. The interface of \compute is as
%% follows:

%% \lgrindfile{code/compute.cpp}

%% Function \v{compute\_init} initializes information about a group of
%% computations developers intend to run in parallel.  \v{group\_size}
%% specifies the number of threads involved.  \v{key} is an optional opaque
%% key identifying the group in case a program has multiple groups of
%% parallel computations going on.  If it is absent, the \v{compute\_init}
%% call refers to a unique anonymous group.  \v{logical\_timeout} is an
%% optional timeout specifying the maximum amount of time \parrot waits for
%% involved threads to arrive at the computations.  This time is not real
%% time.  Instead, it is measured in the number of synchronizations executed
%% by \parrot, so it is always deterministic with respect to the synchronization
%% schedule \parrot enforces.  If \v{logical\_timeout} is absent, \parrot uses a
%% default value of XXX times \v{group\_size}.  Why group size XXX.  We only
%% needed to explicitly set \v{logical\_timeout} for one out of all programs
%% we evaluated.

%% Function \v{compute} indicates that a computation is about to begin.  As
%% in \v{compute\_init}, \v{key} uniquely identifies the computation group.
%% The semantics of \v{compute} is similar to a barrier wait.  When fewer
%% than \v{group\_size} of threads with the same \v{key} have arrived at
%% their respective \v{compute} calls, \parrot blocks the threads until all
%% \v{group\_size} threads arrive or it has executed more than
%% \v{logical\_timeout} synchronizations, most likely avoiding serializing
%% the computations with an unfortunate synchronization schedule.  Different
%% from a barrier wait, \v{compute} is not for synchronizing accesses to
%% shared variables so it can be inaccurate, it is always associated with a
%% timeout, and its timeout is always deterministic.

%% \para{Usage.} \compute hints express developers' high-level parallel
%% computation intents, so below we describe their usage under 
%% four different categories of parallel computation based on the programs we
%% evaluated.  We also note how effective we anticipate the \compute hints to
%% work on different types. Generally we used long sync operation 
%% waiting time to identify the locations of serilization in 
%% code, and we used \vtune to identify the computation amount 
%% of different locations.

%% \begin{tightenum}

%% \item Data partition. The workload data is partitioned based on the
%%   number of threads, and each thread computes on a partition.  Among the
%%   programs we evaluated, \mplayer, \pbzip, all OpenMP programs including
%%   all STL algorithms and \imagick programs, and most of 
%%   benchmark programs (if not mentioend in the other 
%%   categories) fall in this category.  We initialized the group size to the number of threads and
%%   added \v{compute} at the beginning of each thread's local computation.  This
%%   category is the most common, and \compute hints are the most effective
%%   on these programs because the developers strove to partition data evenly
%%   and keep all threads busy.  Therefore, we observed little wait time in
%%   \v{compute} for the threads to arrive and almost no 
%%   timeouts. Note that many programs in this category do not 
%%   need any hints and they already had low overhead 
%%   with our default scheduler. They are \canneal, 
%%   \streamcluster, \barnes, \fft, \luc, \lun, \oceancp, \oceanncp, \watern, 
%%   \waters, \blackscholes.

%% \item Pipeline. The threads are split into stages of a pipeline, and
%%   each item in the workload flows through the pipeline stages.  \ferret,
%%   \dedup, \vips, and \xtwosixfour fall in this category.  The 
%%   methodology of adding \compute hints is to align the 
%%   most performance critical stages in each thread. For 
%%   example, for \vips and \xtwosixfour, we used \vtune to find that all threads are performing 
%%   significant and roughly the same amount of computation 
%%   in each pipeline stage, so we add \compute hints to the 
%%   beginning of thread local computation for all those worker 
%%   threads. And for \ferret, there are six stages, and two 
%%   stages \v{t\_vec} and \v{t\_rank} do two order magnitude more 
%%   computation than all the other stages, so we only need 
%%   to add \compute hints to the two most performance critical stages. 
%%   \dedup has been working well with our default scheduler already.
%%   Interestingly, we observed that most of \compute timeouts 
%%   happened during the periods when the pipeline warmed 
%%   up and cooled down, more details are in \S\ref{sec:sensitivity}.

%% \item Map reduce. Programs in category use both data 
%% partition and pipeline. All the seven mapreduce 
%% programs in \phoenix fall into this category. For all the 
%% sever but one (\wordcount), we only need to add compute hints to the 
%% beginning of each \v{map} function for all worker threads, because 
%% the compuation amount of each \v{reduce} function is 
%% much smaller than the \v{map} function. For \wordcount, we 
%% need \compute hints for both the two functions.

%% \item Workpile.  The workload consists of a pile of independent work
%%  items, processed by worker threads running in parallel.  Among the
%%  programs we evaluated, \bdb, \openldap, \redis, \pfscan, and \aget fall in 
%%  this category. All but one (\pfscan, needs \nondet hints) 
%%  in this category have been working well with our default 
%%  scheduler.

%% \end{tightenum}

%% \noindent
%% Heming: do not need to mention workpile, because they do 
%% not need manual hints.
%% Some programs in other threading models work pretty well with our 
%% default scheduler already. For example, the workpile model.
%%   The workload consists of a pile of independent work
%%   items, processed by worker threads running in parallel.  Among the
%%   programs we evaluated, \bdb, \openldap, \redis, and \aget fall in 
%%   this category.

  %% We found that the \compute hints we added do not 
  %% require 100\% success rate. Among all programs we have 
  %% evaluated, \nproglineuphints programs requrie \compute 
  %% hints in order to get reasonable performance overhead, 
  %% and \nlineupfails of them have less than 100\% sucess 
  %% rate on \compute hints. This case is reasonable because 
  %% in lots of programs, the workload may not be able to be 
  %% evenly divided by the number of threads, so some threads 
  %% tend to handle more computation than others. This fact 
  %% makes the \compute hints suitable to \parrot runtime rather than 
  %% tranditional \pthread barrier, which may corrupt program logic.

  %% We do not need to add \compute hints to solve all seliazation in the programs, 
  %% instead, we only need to solve the serilization in those most serious 
  %% ones (using \vtune or some other tools), and we can already get reasonable performance. 
  %% For exmaple, in \ferret, the pipeline sync operations 
  %% serialize all the six stages, but only two of them is 
  %% significantly more performance-critical than others, so 
  %% we only need to align the two. We also tried to aligh all 
  %% the six stages, but it turns out that the \compute hints 
  %% brought little performance gain.


\subsection{Performance Critical Section} \label{sec:performance-critical-section}

A \vnondet identifies a code region as a potential bottleneck
and encourages the scheduler to get
through the region fast.  When a thread enters a performance critical
section, \parrot removes the thread from the round-robin scheduling 
and delegates it to the OS scheduler for nondeterministic execution.  \parrot thus gains speed
from allowing additional schedules.  The interface is
\vspace{-3 mm}

\lgrindfile{parrot/code/nondet.cpp}
\vspace{-1 mm}

\noindent
The \v{pcs\_enter} function marks the entry of a performance critical section and
\v{pcs\_exit} the exit.

% \parrot treats \v{pcs\_enter} as a special synchronization that
% participates in the round-robin scheduling, so that a performance
% critical section always begins deterministically.  However, \parrot does
% not know how long the section may run, so it lets the section end
% nondeterministically.  After a thread calls \v{pcs\_exit}, \parrot adds it
% back to round-robin.

\subsection{Usage of the Two Hints} \label{sec:hints-usage}

\para{Soft barrier.} Developers should generally use soft barriers to align
high-level, time-consuming parallel computations, such as the \v{compress}
calls in \pbzip. A generic method is to use performance debugging tools or
\parrot's logs to detect synchronizations excessively delayed by \parrot's
round-robin scheduling, then identify the serialized parallel computations.

A second method is to add soft barriers based on parallel computation
patterns.  Below we describe how to do so based on four parallel
computation patterns we observed from the \nprog evaluated programs.

\begin{enumerate}

\item[$\bullet$] {\em Data partition}. Data is partitioned among worker
  threads, and each worker thread computes on a partition. This pattern is
  the most common; \ndatapartition out of the \nprog programs follow this
  pattern, including the programs with fork-join parallelism.  Most
  programs with this pattern need no soft barriers.  In rare cases when
  soft barriers are needed, developers can add \v{soba\_wait} before each
  worker's computation.  These soft barriers often work extremely well.

\item[$\bullet$] {\em Pipeline}. The threads are split into stages of a
  pipeline, and each item in the workload flows through the pipeline
  stages.  \ferret, \dedup, \vips, and \xtwosixfour from \parsec~\cite{parsec} follow this pattern.
  These programs often need soft barriers because
  threads have different roles and thus do different synchronizations,
  causing default schedules to serialize computations.  The
  methodology is to align the most time-consuming stages of the
  pipeline.

\item[$\bullet$] {\em Map-reduce}. Programs with this pattern use both data
  partition and pipeline, so the methodology follows both: align the
  map function and, if the reduce function runs for roughly the same
  amount of time as the map function, align reduce with map.

\item[$\bullet$] {\em Workpile}. The workload consists of a pile of independent
  work items, processed by worker threads running in parallel.  Among the
  programs we evaluated, \bdb, \openldap, \redis, \pfscan, and \aget fall
  in this category.  These programs often need no soft barriers because
  it typically takes similar times to process most items.

\end{enumerate}


\para{Performance critical section.} Unlike a soft barrier, a \nondet may trade
some determinism for performance. Consequently, it should be applied with caution,
only when (1) a code region imposes high performance overhead on deterministic execution
and (2) the additional schedules have been thoroughly checked by tools or
advanced developers.  Fortunately, both conditions are often easy to meet
because the synchronizations causing high performance overhead are often low-level
synchronizations (\eg, lock operations protecting a shared counter),
straightforward to analyze with local reasoning or model checkers.

Of all \nprog evaluated programs, only \nprognondethints need performance critical
sections for reasonable performance; all other \nprognonondethints
programs need not trade determinism for performance.
Moreover, \ecosys verified all schedules in all 4 real-world programs that
need performance critical sections, providing high assurance.

Developers can identify where to add performance critical sections also
using performance debugging tools.  For instance, frequent
synchronizations with medium round-robin delays are often good candidates for
a performance critical section.  Developers can also focus on such
patterns as synchronizations in a tight loop,
synchronizations inside an abstraction boundary (\eg, \v{lock()}
inside a custom memory allocator), and tiny critical sections (\eg,
``\v{lock(); x++; unlock();}'').  

%% A dominant pattern is to mark a tight loop with synchronizations as a
%% performance critical section, so \parrot can pass a thread running in the
%% loop to the OS scheduler for quick finish.

%% \nondet hints are added to tell the \parrot runtime to ignore 
%% some sync operations in perfomance critical 
%% sections, and the per-thread computation in these programs 
%% and not roughly the same (so \compute hints are not 
%% suitable for these cases). 

%% usage: low-level

%% experience: 

%% Nondeterminism hints revert back to nondeterministic execution for
%% carefully chosen code regions.  These regions are often already ``hot''
%% during nondeterministic execution by doing many synchronizations at a high
%% rate, so they cause even very high overhead when executed
%% deterministically.  By moving threads in nondeterministic regions from the
%% round-robin scheduler to the OS's scheduler, \parrot gains speed.


%% \v{nondet(var)} marks all synchronizations on synchronization variable
%% \v{var} as nondeterministic, so \parrot excludes the synchronizations from
%% its round-robin scheduling. Frequently a code region executes many
%% synchronizations touching several synchronization variables, and
%% developers can use

%% \parrot treats \v{nondet\_begin} as a special synchronization that
%% participates in round-robin scheduling, so that a nondeterministic region
%% always begins execution deterministically.  However, \parrot does not know
%% how long the region may run, so it has to let the region end
%% nondeterministically.  After the thread calls \v{nondet\_end}, \parrot adds
%% the thread back to the round-robin scheduling.

%% \para{Usage.} Unlike \compute hints, \nondet hints may trade off 
%% an amount of determinism for performance, so it should be applied with
%% caution, only when (1) a code region has high overhead on deterministic
%% execution and (2) the additional schedules have been thoroughly checked by
%% tools or advanced developers inspecting the code region.  Fortunately,
%% both conditions are often easy to meet because the synchronizations
%% causing high overhead tend to be frequent, low-level synchronizations,
%% easy to analyze with local reasoning. Among all programs 
%% we have evaluated, four real applications \pfscan, three STL programs 
%% \partition, \nthelement, and \partialsort, and five 
%% benchmark programs \fluidanimate, \fmm, 
%% \cholesky, \raytrace from \splashx, and \ua and all fall into this category.
%% For all the four real applications, our ecosystem can 
%% effectively check all these schedules in the \nondet region 
%% within a few hours. For all the five benchmarks programs, 
%% although our ecosystem can not check all their schedules in 
%% \nondet regions within one day's run, we still considered 
%% these results promising because people normally care 
%% about the correctness of real applications over benchmark 
%% programs.

%% Heming: we do not prefer to mention this because this 
%% pattern is not that common in our non-det programs. Only 
%% pfscan and fluidanimate do this.
%% A common pattern we observed on the
%% programs we evaluated is when the program uses a tiny critical section to
%% increment or decrement a shared counter.  It easy to determine based on
%% purely local reasoning that The executions of these critical section are
%% commutable.  This pattern appears in 6 out of 9 programs with \nondet
%% hints.


