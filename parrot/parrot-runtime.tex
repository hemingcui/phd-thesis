\section{\parrot Runtime} \label{sec:parrot-runtime}

%% The simple \parrot runtime consists of a user-space scheduler
%% (\S\ref{sec:scheduler}), a set of wrapper functions for intercepting
%% \pthread synchronizations (\S\ref{sec:sync}), implementations of the hint
%% abstractions (\S\ref{sec:hints-impl}), handlers for network operations
%% (\S\ref{sec:network}), and handlers of timeouts (\S\ref{sec:timeout}).

The \parrot runtime contains implementation of the hint abstractions
(\S\ref{sec:parrot-hints-impl}) and a set of wrapper functions that intercept
\pthread (\S\ref{sec:parrot-sync}), network (\S\ref{sec:parrot-network}), and timeout
(\S\ref{sec:parrot-timeout}) operations.  The wrappers interpose dynamically
loaded library calls via \v{LD\_PRELOAD} and ``trap'' the calls into
\parrot's deterministic scheduler (\S\ref{sec:parrot-scheduler}).  Instead of
reimplementing the operations from scratch, these wrappers leverage
existing runtimes, greatly simplifying \parrot's implementation, deployment,
and inter-operation with code that assumes standard runtimes (\eg,
debuggers).

\subsection{Scheduler} \label{sec:parrot-scheduler}

The scheduler intercepts synchronization calls and releases threads using the
well-understood, deterministic round-robin algorithm: the first thread
enters synchronization first, the second thread second, ..., and
repeat.  It does not control non-synchronization code, often the majority
of code, which runs in parallel.  It maintains a queue
of runnable threads (\emph{run queue}) and another queue of waiting
threads (\emph{wait queue}), like typical schedulers.  Only the head of the
run queue may enter synchronization next. Once the synchronization call is
executed, \parrot updates the queues accordingly.  For instance, for
\v{pthread\_create}, \parrot appends the new thread to the tail of
the run queue and rotates the head to the tail.  By maintaining
its own queues, \parrot avoids nondeterminism in the OS scheduler and
the \pthread library.

\begin{table}[t]
\centering
\small
\begin{minipage}[t]{.25\textwidth}
\lgrindfile{parrot/code/scheduler.cpp}
\end{minipage}
%% \begin{tabular}{l}
%% \v{void get\_turn(void)} \\
%% \v{void put\_turn(void)} \\
%% \v{int wait(void *addr, int timeout)} \\
%% \v{void signal(void *addr)} \\
%% \v{void broadcast(void *addr)} \\
%% \v{void block()} \\
%% \v{void wakeup()} \\
%% \end{tabular}
\vspace{-.05in}
\caption{{\em Scheduler primitives.}} \label{tab:parrot-scheduler}
\vspace{-.05in}
\end{table}

To implement operations in the \parrot runtime, the scheduler provides a
monitor-like internal interface, shown in Table~\ref{tab:parrot-scheduler}.  The
first five functions map one-to-one to functions of a typical monitor,
except the scheduler functions are deterministic.  The last two are for
selectively reverting to nondeterministic execution.  The rest of this
subsection describes these functions.

The
\v{get\_turn} function waits until the calling thread becomes the head
of the run queue, \ie, the thread gets a ``turn'' to do a
synchronization.  The \v{put\_turn} function rotates the calling thread
from the head to the tail of the run queue, \ie, the thread gives up a
turn. The \v{wait} function is similar to
\v{pthread\_cond\_timedwait}.  It requires that the calling thread has the
turn.  It records the address the thread is waiting for and the timeout
(see next paragraph), and moves the calling thread to the tail
of the wait queue.  The thread is moved to the tail of the
run queue when (1) another thread wakes it up via \v{signal}
or \v{broadcast} or (2) the timeout has expired. The \v{wait}
function returns when the calling thread gets a turn again.  Its return
value indicates how the thread was woken up. The \v{signal(void *addr)}
function appends the first thread waiting for \v{addr} to the run queue.  The
\v{broadcast(void *addr)} function appends all threads waiting for
\v{addr} to the run queue in order.  Both \v{signal} and \v{broadcast} require
the turn.

%% The \v{get\_turn()} function waits until the calling thread
%% becomes the head of the run queue, \ie, the thread gets the ``turn'' to do a
%% synchronization. The \v{put\_turn()} function rotates the calling thread from the head
%% to the tail of run queue, \ie, the thread gives up the turn. The \v{wait(void
%%   *addr, int timeout)} function is similar to \v{pthread\_cond\_timedwait}.  It
%% requires that the calling thread has the turn.  It records the address the
%% thread is waiting for and the timeout (\cf next paragraph) of the wait, and
%% moves the calling thread to the tail of the wait queue.  The thread is woken
%% up and moved to the tail of the run queue when (1) another thread wakes it up
%% via the \v{signal} function or the \v{broadcast} function or (2) the timeout has 
%% expired. The \v{wait} function returns when the calling thread gets the turn again.  Its return value
%% indicates how the thread is woken up. The \v{signal(void *addr)} function searches
%% wait queue for the first thread waiting for \v{addr} and appends it to run
%% queue. The \v{broadcast(void *addr)} function wakes up all threads waiting for \v{addr}
%% and appends them to run queue in order.  Both the \v{signal} 
%% and the \v{broadcast} functions require the turn.

The \v{timeout} in the \v{wait} function does not specify real time, but relative \emph{logical time} that
counts the number of turns executed since the beginning of current
execution.  In each call to the \v{get\_turn} function, \parrot increments this logical
time and checks for timeouts. 
(If all threads block, \parrot keeps the logic time advancing with an idle
thread; see~\S\ref{sec:parrot-timeout}.)
The \v{wait} function takes a relative timeout argument.  If
current logical time is $t_l$, a timeout of 10 means waking up the thread
at logical time $t_l + 10$. A \v{wait(NULL, timeout)} 
call is a logical sleep, and a \v{wait(addr, 0)} call never times out.
% Logical time makes it easy to implement many operations real-world
% programs rely on (\S\ref{sec:timeout}).

\begin{figure}[t]
%\centering
\hspace{1.5in}
\begin{minipage}[t]{.38\textwidth}
\lgrindfile{parrot/code/lock.cpp}
\end{minipage}
\vspace{-1 mm}
\caption{{\em Wrappers of \pthread mutex lock\&unlock.}} \label{fig:parrot-lock}
\vspace{-0.0in}
\end{figure}

The last two functions in Table~\ref{tab:parrot-scheduler} support performance
critical sections and network operations.  They set the calling thread's
execution mode to nondeterministic or deterministic. \parrot always schedules
synchronizations of deterministic threads using round-robin, but it lets
the OS scheduler schedule nondeterministic threads.
Implementation-wise, the \v{nondet\_begin} function marks the calling
thread as nondeterministic and simply returns.  This thread will be lazily
removed from the run queue by the thread that next tries to pass the turn to it.
(Next paragraph explains why the lazy update.)
The \v{nondet\_end} function marks the calling thread as deterministic and
appends it to an additional queue.  This thread will be lazily appended to
the run queue by the next thread getting the turn.

We have optimized the multicore scheduler implementation for the most frequent
operations: \v{get\_turn}, \v{put\_turn}, \v{wait}, and \v{signal}.  Each
thread has an integer flag and condition variable. The \v{get\_turn} function
spin-waits on the current thread's flag for a while before block-waiting
on the condition variable. The \v{wait} function needs to get the turn before it
returns, so it uses the same combined spin- and block-wait strategy as
the \v{get\_turn} function. The \v{put\_turn} and the 
\v{signal} functions signal both the flag and the
condition variable of the next thread.  In the common case, these
operations acquire no lock and do not block-wait.  The lazy updates above
simplify the implementation of this optimization by maintaining the
invariant that only the head of the run queue can modify the run and wait
queues.

%% In order to support \nondet operations such as I/O operations, 
%% we introduce two more schedule primitives. The first one is 
%% \v{block()} called by a thread before doing a \nondet 
%% operation. It gets turn, removes it self from \emph{run queue}, 
%% and then passes the turn to the thread in the head of the \emph{run queue}. 
%% This operation ensures that  while a thread is doing a \nondet operation,
%% it self is not in the \emph{run queue} of \parrot, so that the \parrot runtime 
%% does not enforce any order between it and other threads within the runtime 
%%  (deadlock-free). The second one \v{wakeup()} is called 
%%  after a thread gets back from a \nondet opeartion. It appends the thread 
%%  itself to a \emph{wakeup queue} , an extra data structure for \nondet 
%%  operations. In \parrot, after each thread gets its turn, it  will check this 
%%  \emph{wakeup queue} and put the threads from this queue back to
%%  \emph{runq queue}. A thread getting back from \nondet operations to
%% \parrot runtme is not allowed to put itself back to the \emph{runq 
%% queue}  directly, because our \parrot runtime ensures that only the 
%% head of \emph{runq queue} is allowed to modify the \emph{runq queue}.

\subsection{Synchronizations} \label{sec:parrot-sync}

%% The wrapper functions intercept \pthread synchronizations by interposing
%% dynamically loaded library calls (via \v{LD\_PRELOAD}) and ``trapping''
%% them into the scheduler.  Wrappers are simple.  Instead of reimplementing
%% \pthread synchronizations from scratch, they leverage existing \pthread
%% runtimes, greatly simplifying the code, deployment, and inter-operation
%% with code that assumes standard \pthread runtimes (\eg, debuggers).
% Wrappers replicate a small amount of control data in the \pthread
% runtime, such as the size of a barrier and the locked status of a mutex,
% to determine whether the intercepted synchronizations block.
%% Wrappers ensure a total (round-robin) order of synchronizations by (1)
%% using the scheduler primitives to ensure that at most one wrapper has the
%% turn and (2) executing the actual synchronizations only when the turn is held.

% To do so, the first thing a wrapper does is to wait for its turn, \ie,
% the calling thread becoming the head of the queue, and the last thing is
% to give up the turn by moving the calling thread to the tail of the run
% or wait queue.

\parrot handles all synchronizations on \pthread mutexes, read-write
locks, condition variables, semaphores, and barriers. It also handles thread
creation, join, and exit.  It need not implement the other \pthread
functions such as thread ID operations, another advantage of leveraging
existing \pthread runtimes. In total, \parrot has \npthreadsync synchronization
wrappers.  They ensure a total (round-robin) order of synchronizations by
(1) using the scheduler primitives to ensure that at most one wrapper has
the turn and (2) executing the actual synchronizations only when the turn
is held.

%  Below we illustrate how we implemented the \parrot runtime for three example synchronizations.
% (Interested readers may refer to \parrot's source code~\cite{parrot-github}
% for others.)

% \subsubsection{Locks}

Figure~\ref{fig:parrot-lock} shows the pseudo code of our \pthread mutex lock and
unlock wrappers.  Both are quite simple; so are most other wrappers.  The
lock wrapper uses the try-version of the \pthread lock operation to avoid
deadlock: if the head of run queue is blocked waiting for a lock before
giving up the turn, no other thread can get the turn.

%% For \v{pthread\_mutex\_lock} but the lock is already held,
%% \parrot moves the current thread from the head of the run queue to the tail
%% of the wait queue and record the address of the mutex the thread is
%% waiting for.  For \v{pthread\_mutex\_unlock}, \parrot wakes up the first
%% thread waiting for the mutex if there is one from the wait queue.  After
%% running the synchronizations, \parrot rotates the current thread to the end
%% of the run queue if the synchronization did not cause it to block on the
%% wait queue.  

%% The unlock wrapper.

% \subsubsection{Condition Variables} \label{sec:cond_wait}

\begin{figure}[t]
\centering
\begin{minipage}[t]{.5\textwidth}
\lgrindfile{parrot/code/cv.cpp}
\end{minipage}
\vspace{-2 mm}
\caption{{\em Wrapper of \v{pthread\_cond\_wait}.}} \label{fig:parrot-cv}
\vspace{-0 mm}
\end{figure}

Figure~\ref{fig:parrot-cv} shows the \v{pthread\_cond\_wait}
wrapper.  It is slightly more complex than the lock and unlock wrappers
for two reasons.  First, there is no try-version of
\v{pthread\_cond\_wait}, so \parrot cannot use the same trick to avoid
deadlock as in the lock wrapper.  Second, \parrot must ensure that unlocking
the
mutex and waiting on the conditional variable are atomic (to avoid the
well-known lost-wakeup problem).  \parrot solves these issues by implementing
the wait with the scheduler's \v{wait} which atomically gives up the
turn and blocks the calling thread on the wait queue.  The wrapper of
\v{pthread\_cond\_signal} (not shown) calls the scheduler's \v{signal}
accordingly.

%% which blocks the calling thread regardless, \parrot must release the
%% scheduler lock before the blocking to avoid deadlock.  Yet, if \parrot lets
%% the synchronization run outside of

%% \subsubsection{Thread Creation}

%% \begin{figure}[t]
%% \centering
%% \begin{minipage}[t]{.5\textwidth}
%% \lgrindfile{code/pthread_create.cpp}
%% \end{minipage}
%% \caption{{\em Wrapper of \v{pthread\_create}.}} \label{fig:create}
%% \end{figure}

%% \begin{figure}[t]
%% \begin{verbatim}
%% two threads calling pthread_create, two new threads
%% first pthread\_create thread wrongly pairs up
%%    with last new thread
%% \end{verbatim}
%% \caption{{\em \v{pthread\_create} race.}} \label{fig:create-race}
%% \end{figure}

% Figure~\ref{fig:create} shows the pseudo code of the \v{pthread\_create}
% wrapper.

Thread creation is the most complex of all wrappers for two reasons.
First, it must deterministically assign a logical thread ID
to the newly created thread because the system's thread IDs are
nondeterministic.  Second, it must also prevent the new thread from using
the logical ID before the ID is assigned.  \parrot solves these issues by
synchronizing the current and new threads with two semaphores, one to make
the new thread wait for the current thread to assign an ID, and the other to
make the current thread wait until the child gets the ID.

%% Specifically, \v{wrap\_thread\_create} waits for the turn, saves the
%% user-provided thread function, and calls the actual \v{pthread\_create} to
%% create a thread to run \v{wrap\_thread\_func}, a wrapper to the
%% user-provided thread function.  Function \v{wrap\_thread\_create}
%% continues by recording the mapping from the new thread's \pthread ID to
%% logical thread ID, and adds the new thread to run queue.  It then
%% wakes up the new thread blocked inside \v{wrap\_thread\_func}, waits for
%% the new thread to get its logical thread ID, and finally gives up the
%% turn.  Both semaphores are necessary.  Function \v{wrap\_thread\_func}
%% must wait for \v{wrap\_thread\_create} to set up the mapping from the
%% \pthread ID to logical ID.  Function \v{wrap\_thread\_create} must wait for
%% \v{wrap\_thread\_func} before giving the turn because otherwise another
%% newly created thread may be wrongly woken up.
%, as illustrated in Figure~\ref{fig:create-race}.

%% One complication is the wrapper of \v{pthread\_cond\_wait(cond, mutex)}.
%% This synchronization blocks the calling thread on condition variable
%% \v{cond} regardless, so \parrot must release the scheduler lock before the
%% blocking to avoid deadlock.  However, the semantics also requires that
%% unlocking \v{mutex} and waiting on \v{cond} be atomic (to avoid the
%% lost-wakeup problem).

\subsection{Performance Hints} \label{sec:parrot-hints-impl}

%% \begin{figure}[t]
%% \centering
%% \begin{minipage}[t]{.5\textwidth}
%% \lgrindfile{code/compute-algo.cpp}
%% \end{minipage}
%% \caption{{\em Pseudo code of soft barrier.}} \label{fig:compute}
%% \end{figure}

%% Figure~\ref{fig:compute} shows the implementation of soft barrier, a
%% reusable barrier with a deterministic timeout.  \parrot implements

\parrot implements performance hints using the scheduler primitives.  It
implements the soft barrier as a reusable barrier with a deterministic
timeout.  It implements the performance critical section by simply calling
\v{nondet\_begin()} and \v{nondet\_end()}.
% to temporarily enable nondeterministic execution of a thread within a
% performance critical region.

One tricky issue is that deterministic and nondeterministic executions may
interfere.  Consider a deterministic thread $t_1$ trying to lock a mutex
that a nondeterministic $t_2$ is trying to unlock.
Nondeterministic thread $t_2$ always ``wins'' because the timing of
$t_2$'s unlock directly influences $t_1$'s lock regardless of how hard
\parrot tries to run $t_1$ deterministically.  An additional concern is
deadlock: \parrot may move $t_1$ to the wait queue but never wake $t_1$ up
because it cannot see  $t_2$'s unlock.

To avoid the above interference, \parrot requires that synchronization variables accessed
in nondeterministic execution are isolated from those accessed in
deterministic execution.  This \emph{strong isolation} is easy to achieve based
on our experiments because, as discussed in~$\S$\ref{sec:parrot-hints}, the
synchronizations causing high overhead on deterministic execution tend to
be low-level synchronizations already isolated from other
synchronizations. To help developers write performance critical sections
that conform to strong isolation, \parrot checks this property at runtime:
it tracks two sets of synchronization variables accessed within
deterministic and nondeterministic executions, and emits a warning when the
two sets overlap.  Strong isolation is considerably stronger than
necessary: to avoid interference, it suffices to forbid
deterministic and nondeterministic sections from \emph{concurrently}
accessing the same synchronization variables.  We have not implemented
this \emph{weak isolation} because strong isolation works well for all
programs evaluated.

%%  before each \nondet operation.  Although this
%% approach is crude, we find it pretty lightweight and the warnings are
%% already quite helpful to us when adding \nondet hints.

%% One concern with our implementation of \nondet hints is deadlock.  The
%% system scheduler runs all threads in nondeterministic regions, whereas
%% \parrot runs all other threads.  The system scheduler and \parrot may attempt to
%% enforce contradictory constraints, causing deadlocks like in
%% Figure~\ref{fig:deadlock}.

%% We sketch a proof that our implementation of performance critical section
%% does not introduce deadlocks.  Suppose a program has no deadlock running
%% with either the system's scheduler or \parrot, but it has a deadlock when
%% running with both.  There are two scenarios.  First, the head of run queue
%% is blocked without giving up the turn. The blocking synchronization must
%% be done by a nondeterministic thread, or \parrot would have noticed.
%% However, this scenario is impossible because a nondeterministic thread
%% cannot be the head of run queue.  Second, some thread on wait queue should
%% have been signaled but was not.  The lost signal must be done by a
%% nondeterministic thread, but this scenario is also impossible because
%% synchronization variables are strongly isolated.


%% The \compute hint works in a similar way as \pthread 
%% barrier, except it has a deterministic logical timeout. 
%% When a thread fires a timeout for this hint, all the other 
%% threads blocking on this hint will be waken up, get turn and then proceed.
%% This deterministic logical timeout ensures that \compute 
%% hint does not affect the correctness of a program and the 
%% execution is still deterministic.

%% The \v{nondet\_begin()} hint removes the current thread from \emph{run 
%% queue}, and let the system scheduler run it. When the 
%% threa gets back, the \v{nondet\_end()} hint add it self 
%% back to the \emph{wakeup queue}, and the next thread 
%% getting the turn will put all threads from \emph{wakeup 
%% queue} back to \emph{run queue}. The system scheduler schedules 
%% all threads in nondet region.  \parrot schedulers all other threads.

%% Mixing deterministic and nondeterministic executions is tricky because
%% they can interfere.  Consider two threads: $t_1$ in deterministic
%% execution trying to lock a mutex and $t_2$ in nondeterministic execution
%% trying to unlock the same mutex.  The nondeterministic thread $t_2$ always
%% ``wins'' because regardless of how hard \parrot tries to run $t_1$
%% deterministically, the timing of $t_2$'s unlock directly influences
%% $t_1$'s lock.  In general, each nondeterministic synchronization may
%% interfere with deterministic execution.  As nondeterministic
%% synchronizations accumulate, their effects cascade boundlessly, ruining
%% deterministic execution.  Worse, since a code region may execute many
%% synchronizations, a single pair of \v{nodet\_begin} and \v{nondet\_end}
%% may have a drastic effect on determinism, extremely counterintuitive to
%% developers.

%% To avoid interference, we have carefully designed the semantics of \nondet
%% hints to strongly isolate synchronization variables accessed in
%% nondeterministic execution from those accessed in deterministic execution,
%% which \parrot checks at run time (\S\ref{sec:hints-impl}).  This strong
%% isolation turned out to be pretty easy to achieve on the programs we
%% evaluated because the synchronizations causing high overhead on
%% deterministic execution tend to be frequent, low-level synchronizations
%% that are already isolated from other synchronizations (\eg,
%% synchronizations protecting the increment of a shared counter).

%% Strong isolation has three nice features.  First, it bounds the effects of
%% the execution of a nondeterministic region on deterministic
%% execution---regardless of how many synchronizations are run, the only
%% effect is the nondeterministic execution ending.  Second, it vastly
%% simplifies the implementation of \nondet hints, to the extent that we
%% proved that our implementation is deadlock-free (\S\ref{sec:hints-impl}).
%% Third, it greatly reduces the number of executions a model checker has to
%% check because the nondeterministic synchronizations do not affect
%% deterministic executions.

%% In order to assist users to ensure strong isolation of synchronization
%% variables accessed in deterministic and nondeterministic 
%% regions, \parrot simply tracks the variables during their life time with two sets.
%% For each synchronization, it adds the variable to the
%% corresponding set protected by an internal spin lock. \parrot emits warnings
%% when the two sets overlap before each \nondet operation.
%% Although this approach is crude, we find it pretty lightweight
%% and the warnings are already quite helpful to us when adding \nondet hints.

%% One concern with our implementation of \nondet hints is deadlock.  The
%% system scheduler runs all threads in nondeterministic regions, whereas
%% \parrot runs all other threads.  The system scheduler and \parrot may attempt to
%% enforce contradictory constraints, causing deadlocks like in
%% Figure~\ref{fig:deadlock}.

%% We present a proof sketch that our \nondet hints do not introduce
%% deadlocks.  Suppose a program has no deadlock running with either the
%% system's scheduler or \parrot, but it has a deadlock when running with both.
%% There are two scenarios.  First, the head of the queue is blocked without
%% giving up the turn. The blocking synchronization must be inside a
%% nondeterministic region, like XXX in Figure~\ref{fig:deadlock}, or \parrot
%% would have noticed.  However, this scenario is impossible because a thread
%% running in a nondeterministic region is removed from \parrot's run queue and
%% cannot be the head of the queue.  Second, some thread on \parrot's wait queue
%% should have been signaled but was not.  The lost signal must be inside a
%% nondeterministic region, but this scenario is also impossible because
%% synchronization variables are strongly isolated.

%% We present a proof sketch that our \nondet hints do not introduce
%% deadlocks.  Suppose a program has no deadlock running with either the
%% system's scheduler or \parrot, but it has a deadlock when running with both.
%% Consider the \emph{wait cycle} of the deadlock, a graph with
%% synchronizations as vertexes and edges from synchronization $s_1$ to $s_2$
%% if $s_1$ can run only after $s_2$.  Figure~\ref{fig:deadlock} is an
%% example.  We use dotted lines to indicate the edges enforced by \parrot for
%% round-robin, and solid lines for those due to synchronization semantics.
%% The wait cycle of the deadlock must involve at least one synchronization
%% blocked by \parrot for round-robin and another blocked by the system's
%% scheduler due to synchronization semantics.  (Otherwise, the program has a
%% deadlock with either the system's scheduler or \parrot.)  The path from the
%% latter synchronization to the former in the wait cycle must have an edge
%% from a synchronization to a deterministic one.  Due to strong isolation,
%% these two synchronizations must be in the same thread, like XX and XX in
%% Figure~\ref{fig:deadlock}.  However, this edge is impossible because a
%% thread running in a nondeterministic region, 

%% by the system's scheduler waiting be woken up by another
%% synchronization. Otherwise, the program has a deadlock with either the
%% system's scheduler or \parrot.  The path from the

%% one synchronization
%% blocked by \parrot for the turn and another blocked by the system's scheduler
%% waiting be woken up by another synchronization.  

\subsection{Network Operations} \label{sec:parrot-network}

To handle network operations, \parrot leverages the \v{nondet\_begin} and
\v{nondet\_end} primitives.  Before a blocking operation such as \v{recv},
it calls \v{nondet\_begin} to hand the thread to the OS scheduler.  When
the operation returns, \parrot calls \v{nondet\_end} to add the thread back to
deterministic scheduling.  \parrot supports \nioync network
operations such as \v{send}, \v{recv}, \v{accept}, and \v{epoll\_wait}.
This list suffices to run all evaluated programs that require network
operations (\bdb, \openldap, \redis, and \aget).

\subsection{Timeouts} \label{sec:parrot-timeout}

Real-world programs use timeouts (\eg, \v{sleep},
\v{epoll\_wait}, and \v{pthread\_cond\_timedwait}) for periodic activities
or timed waits.  Not handling them can lead to nondeterministic execution and
deadlocks.  One deadlock example in our evaluation was running \pbzip with
\dthreads: \dthreads ignores the timeout in
\v{pthread\_cond\_timedwait}, but \pbzip sometimes relies on the timeout to finish.

\parrot makes timeouts deterministic by proportionally converting them to a
logical timeout.  When a thread registers a relative timeout that fires
$\Delta t_r$ later in real time, \parrot converts $\Delta t_r$ to a relative
logical timeout $\Delta t_r /R$ where $R$ is a configurable conversion
ratio. ($R$ defaults to 3 $\mu$s, which works for all evaluated programs.)
Proportional conversion is better than a fixed logical timeout because it
matches developer intents better (\eg, important activities
run more often).  A nice fallout is that it makes some
non-terminating executions terminate for model checking
(\S\ref{sec:parrot-coverage}).  Of course, \parrot's logical time corresponds
loosely to real time, and may be less useful for real-time applications.\footnote{
dOS~\cite{dos:osdi10} discussed the possibility of converting real time to
logical time but did not present how.}

%% \parrot makes timeouts deterministic by proportionally converting them to a
%% logical timeout.  When a thread registers a relative timeout that fires
%% $\Delta t_r$ later in real time, \parrot converts $\Delta t_r$ to a relative
%% logical timeout $\Delta t_r /R$ where $R$ is a configurable conversion
%% ratio (default is 3 $\mu$s), which worked for all evaluated programs.
%% Proportional conversion is better than a fixed logical timeout because it
%% matches developer intents better.  For instance, developers may want to
%% perform one important periodical activity with higher frequency, and a
%% less important one with lower frequency.  Proportional conversion keeps
%% the relative ratio of the frequencies, and improves performance based on
%% our experiments. A nice fallout is that it makes some non-terminating
%% executions terminate for model checking (\S\ref{sec:coverage}).
%% Of course, \parrot's logical time corresponds loosely to
%% real time, and may be less useful for real-time applications.
%% (dOS~\cite{dos:osdi10} discussed the possibility of converting real to
%% logical time but did not present how.)

When all threads are on the wait queue, \parrot spawns an idle thread to keep the
logical time flowing. The thread repeatedly gets the turn, sleeps for time
$R$, and gives up the turn.  An alternative to idling is
fast-forwarding~\cite{modist:nsdi09,dos:osdi10}.  Our experiments show
that using an idle thread has better performance than fast-forwarding
because the latter often wakes up threads prematurely before the pending
external events (\eg, receiving a network packet) are done, wasting CPU cycles.

%% Moreover, it fast-forwards logical time after all threads block, slower
%% than using an idle thread.

\parrot handles all such common operations as \v{sleep} and
\v{pthread\_cond\_timedwait}, enough for all five evaluated programs
that require timeouts (\pbzip, \bdb, \mplayer, \imagick, and \redis).
\pthread timed synchronizations use absolute time, so \parrot provides
developers a function \v{set\_base\_time} to pass in the base time.  It
uses the delta between the base time and the absolute time argument as $\Delta
t_r$.
% The following programs we evaluated require timeout support: .

%% We find supporting these operations crucial, because lots
%% of real applications such as \pbzip, \bdb, \mplayer, \imagick and \redis
%% use them. One main reason that we could not make \dthreads and \coredet
%% work with any real application in our benchmarks during performance
%% comparison (Figure~\ref{fig:comparison}) is they did not support these
%% operations.






