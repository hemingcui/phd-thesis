\section{\xxx-\dbug Ecosystem} \label{sec:mc}

%% In this section, we present an ecosystem formed by integrating \xxx with
%% the model checker \dbug~\cite{dbug:spin11}.  We present some background and rationale
%% (\S\ref{sec:mc-background}), briefly introduce \dbug (\S\ref{sec:dbug}),
%% and describe the integration (\S\ref{sec:smt+mc}).

%% \subsection{Background and Rationale} \label{sec:mc-background}

Model checking is a formal verification technique that systematically
explores possible executions of a program for bugs.  These executions
together form a \emph{state space} graph, where states are snapshots of the
running program and edges are nondeterministic events that move the
execution from one state to another.  This state space is typically very
large, impossible to completely explore---the so-called \emph{state-space
  explosion} problem.  To mitigate this problem, researchers have created
many heuristics~\cite{yang:fisc:osdi,musuvathi:aodv,killian:macemc:nsdi07} to guide the exploration toward executions
deemed more interesting, but heuristics have a risk of missing bugs.
\emph{State-space reduction} techniques~\cite{flanagan:dynamicpo,godefroid:verisoft,demeter:sosp11} soundly prune
executions without missing bugs, but the effectiveness of these techniques
is limited.  They work by discovering equivalence: given
that execution $e_1$ is correct if and only if $e_2$ is, we need check only
one of them. Unfortunately, equivalence is rare and extremely challenging
to find, especially for \emph{implementation-level} model checkers which
check implementations directly~\cite{godefroid:verisoft,musuvathi:aodv,yang:fisc:osdi,yang:explode:osdi,killian:macemc:nsdi07,dbug:spin11}.
This difficulty is reflected in the existence of only two main reduction
techniques~\cite{flanagan:dynamicpo, demeter:sosp11} for these implementation-level model
checkers.  Moreover, as a checked system scales, the state space after
reduction still grows too large to fully explore.  Despite
decades of efforts, state-space explosion remains the bane of model
checking.

As discussed in~$\S$\ref{sec:intro}, integrating \smt and model checking is
mutually beneficial.  By reducing schedules, \smt offers an extremely
simple, effective way to mitigate and sometimes completely solve the
state-space explosion problem without requiring equivalence.  For
instance, \xxx enables \dbug to verify \nprogverifiedxxx programs,
including 4 programs containing \nondets (\S\ref{sec:coverage}).
In return, model
checking helps check the schedules that matter for \xxx and developers.
For instance, it can check the default schedules chosen by \xxx, the
faster schedules developers choose using \computes, or the schedules
developers add using \nondets.

%% One interesting idea is to
%% leverage these mutual benefits to release software early with a small set
%% of tested schedules, then gradually increase the schedules for performance
%% after they have been model checked. We leave this idea to future work.

%% \smt offers an extremely simple, effective way to reduce the
%% state space by reducing many schedules.  It then enforces the checked
%% schedules at runtime, greatly increasing model checking coverage.  We call
%% this approach what-you-check-is-what-you-run.  On the flip side, model
%% checking can check the schedules that matter to \smt.

\subsection{The \dbug Model Checker} \label{sec:dbug}

In principle, \xxx can be integrated with many model checkers.  We chose
\dbug~\cite{dbug:spin11} for three reasons.  First, it is 
open source, checks implementations directly, and 
supports \pthread synchronizations and Linux 
socket calls.  Second, it implements one of the
most advanced state-space reduction techniques---dynamic partial order
reduction (DPOR)~\cite{flanagan:dynamicpo}, so the further reduction \xxx
achieves is more valuable.  Third, \dbug can estimate the size of the
state space based on the executions explored, a technique particularly
useful for estimating the reduction \xxx can achieve when
the state space explodes.

Specifically, \dbug represents the state space as an \emph{execution tree}
where nodes are states and edges are choices representing the operations
executed.  A path leading from the root to a leaf encodes a unique test
execution as a sequence of nondeterministic operations.  The total number
of such paths is the state space size.  To estimate this size based on
a set of explored paths, \dbug uses the \emph{weighted backtrack
  estimator}~\cite{Kilby2006}, an online variant of Knuth's offline
technique for tree size estimation~\cite{Knuth1975}.  It treats the set of
explored paths as a sample of all paths assuming uniform distribution over
edges, and computes the state space size as the number of explored
paths divided by the aggregated probability they are explored.

%% Formally, every time a path is
%% explored, WBE updates the estimate to:
%% \[
%% \mathit{estimate} =
%% %\frac{\sum\limits_{b \in B} p(b) * p^{-1}(b)}{\sum\limits_{b \in B} p(b)} =
%% \frac{|B|}{\sum\limits_{b \in B} p(b)}
%% \]
%% where $B$ is the set of explored paths, and $p(b)$ is the probability of
%% exploring the path. For a path $b = (v_1, \ldots, v_n)$, where $v_i$ is
%% the $i$-th node along the path $b$, the probability $p(b)$ is defined
%% as: \[\prod\limits_{i = 1}^{n - 1} \frac{1}{E(v_i)}\] where $E(v_i)$ is
%% the number of edges originating from $v_i$.

%% We have integrated \xxx with a model checker called
%% \dbug~\cite{dbug:spin11}.  We chose \dbug over other model checkers for
%% three reasons.

%% First, it runs in Linux and readily supports Pthreads.
%% Second, \dbug implements some of the most advanced state-space reduction
%% techniques such as dynamic partial order reduction~\cite{popl:dpor}, so
%% the further reduction \xxx can achieve is more valuable.  Third, \dbug
%% provides a technique to estimate the state space size based on the
%% executions explored~\cite{xxx}.  Given that model checkers often cannot
%% fully explore the state space, \dbug's estimated state space size help us
%% compute an estimated reduction \xxx can achieve.

%% \subsubsection{State Space Exploration}
\dbug~\cite{dbug:spin11} is a tool for systematic testing of concurrent
Linux programs. The goal of \dbug is to explore the state space of
different program states of a concurrent program by systematically
enumerating different orders in which concurrent POSIX threads and
Linux sockets calls can be exercised by a program test.

To this end, \dbug abstractly represents the state space of a program
test using an \emph{execution tree}. Nodes of the execution tree
represent non-deterministic choice points and edges record
non-deterministic choices representing program state transitions. A
branch leading from the root of the tree to a leaf thus encodes a
unique test execution as a sequence of non-deterministic scheduling
choices.

\dbug uses this abstraction to repeatedly execute a program test,
recording what sequence of program state transition have been explored
by each iteration and guaranteeing that different iterations explore
different sequences.

\subsubsection{State Space Reduction}
To mitigate the combinatorial explosion resulting from having to
consider all possible permutations of concurrently enabled
transitions, \dbug implements dynamic partial order
reduction (DPOR)~\cite{flanagan:dynamicpo}.

A key building block of DPOR is the \emph{independence} relation that
identifies pairs of commutitative program state transitions. To avoid
the prohibitive overhead of having to track each memory access to
compute the independence relation precisely, \dbug adopts the approach
used by RacePro~\cite{racepro:sosp11}. In particular, for each program state
transition, \dbug records what abstract resources this transition
accesses, and uses this information to approximate the independence
relation.

\subsubsection{State Space Estimation}
To estimate the progress of the exploration, \dbug uses the
\emph{weighted backtrack estimator} (WBE)~\cite{Kilby2006} to estimate
the size of a partially explored execution tree. WBE is an online
variant of Knuth's offline technique for tree size
estimation~\cite{Knuth1975}. WBE uses the length of each explored branch
weighted by the probability it is explored (assuming uniform
distribution over edges) to predict the size of the tree. Formally,
WBE updates the estimate every time a branch of the execution is
explored, setting the estimate to:
\[
\mathit{estimate} =
\frac{\sum\limits_{b \in B} p(b) * p^{-1}(b)}{\sum\limits_{b \in B} p(b)} =
\frac{|B|}{\sum\limits_{b \in B} p(b)}
\]
where $B$ is the set of explored branches, and $p(b)$ is the
probability of exploring the branch. For a branch $b = (v_1, \ldots,                                                                                          
v_n)$, where $v_i$ is the $i$-th node along the branch $b$, the
probability $p(b)$ is defined as: \[\prod\limits_{i = 1}^{n - 1}
\frac{1}{E(v_i)}\] where $E(v_i)$ is the number of enabled
edges originating from node $v_i$.

\subsubsection{Source Code Annotations}
In general, \dbug does not require any changes to the source code. In
fact, as \dbug uses run-time interposition to control the order in
which concurrent POSIX threads and Linux sockets calls happen, it does
not require access to the source code at all.  However, similar to the
SysNam annotations, in some cases, source code annotations can improve
the efficiency of \dbug.

For example, a programmer can use the {\tt \dbug\_off()} and {\tt
  \dbug\_on()} annotations to exclude parts of the tested program from
the scope of \dbug, avoiding exploration of uninteresting call
interactions. (Example: using a lock to increment a variable in
pfscan). Further, a programmer can use the {\tt int \dbug\_nondet(int
  event)} function to identify a non-deterministic event. The first
time such an event occurs, \dbug records its value and makes sure the
same value is used for the same event in future iterations. (Example:
Recording the return values of {\tt gettimeofday()} for redis).
Finally, a programmer can use the {\tt int \dbug\_choose(int n)}
function to non-determinstically choose a value between $0$ and
$n-1$. (Example: Simulating timeouts or system call failures).


\subsection{Integrating \xxx and \dbug} \label{sec:smt+mc}

A key integration challenge is that both \xxx and \dbug control the order
of nondeterministic operations and may interfere, causing
difficult-to-diagnose false positives.  A na\"{i}ve solution
is to replicate \xxx's scheduling algorithm inside \dbug.  This
approach is not only labor-intensive, but also risky because the
replicated algorithm may diverge from the real one, deviating the checked
schedules from the actual ones.

Fortunately, the integration is greatly simplified because performance
critical sections make nondeterminism explicit, and \dbug can ignore
operations that \xxx runs deterministically.  \xxx's strong-isolation
semantics further prevent interference
between \xxx and \dbug.  Our integration uses a nested-scheduler
architecture similar to Figure~\ref{fig:arch} except the
nondeterministic scheduler is \dbug.  This architecture is transparent
to \dbug, and requires only minor changes (\locsmcmc lines) to \xxx.
First, we modified \v{nondet\_begin} and \v{nondet\_end} to turn \dbug
on and off.  Second, since \dbug explores event orders only after
it has received the full set of concurrent events, we modified
\xxx to notify \dbug when a thread transitions between the run queue
and the wait queue in \xxx. These notifications help \dbug accurately
determine when all threads in the system are waiting for \dbug to make
a scheduling decision.

We found two pleasant surprises in the
integration.  First, \computes speed up \dbug executions.  Second,
\xxx's deterministic timeout (\S\ref{sec:timeout}) prevents \dbug from
possibly having to explore infinitely many schedules.  Consider the
``\v{while(!done) sleep(30);}'' loop which can normally
nondeterministically repeat any number of times before making
progress.  This code has only one schedule with \ecosys because \xxx
makes the \v{sleep} return deterministically.

%% that runs \xxx inside
%% a model checker.  \xxx deterministically schedules all synchronizations
%% outside of nondeterministic regions, so it hides them from the model
%% checker.  It exposes the events in nondeterministic regions to the model
%% checker, which can then systematically explore different orders of these
%% events.  This architecture is transparent to a model checker, enabling
%% \xxx to leverage whatever model checkers~\cite{xxx}, advanced search
%% heuristics, and state-space reduction techniques~\cite{xxx}.  It also
%% requires little change to \xxx to forward the nondeterministic events to
%% the model checker.

%% To integrate \xxx with \dbug, we made no change to \dbug and only two
%% minor changes to \xxx.  First, we modified \v{nondet\_begin} and
%% \v{nondet\_end} to attach and detach a thread to \dbug, so that \dbug can
%% track the events executed in nondeterministic regions.  Second, since
%% \dbug explores event orders only after it has received the full set of
%% concurrent events, we modified \xxx to block threads in \v{nondet\_end}
%% until all threads, including the deterministic ones, are blocked, so it
%% can accumulate a maximal set of concurrent events and call into \dbug to
%% explore them.


