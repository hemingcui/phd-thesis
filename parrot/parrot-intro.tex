The last two chapters have presented our two systems, \tern and \peregrine,  
with each addressing a distinct challenge on building \smt. Other 
researchers have also been building and applying \smt 
systems~\cite{determinator:osdi10, dthreads:sosp11, bergan:oopsla13} to improve 
software reliability. However, despite these latest advances, existing \smt 
systems either require sophisticated program analysis (\eg, our two previous 
systems), or incur prohibitive performance overhead (\eg, a previous 
system~\cite{dthreads:sosp11} incurred 30$\times$ slowdown with many programs 
in our evaluation in \S\ref{sec:parrot-eval}), causing \smt difficult to be 
widely deployed. Thus, a third challenge on building \smt arises: how to make 
\smt simple, fast, and deployable? To address this challenge, this chapter 
presents \parrot, a simple, deployable \smt runtime system, and its 
novel programming abstraction called \emph{performance hints} that make 
\parrot's schedules run fast.

\section{Introduction} \label{sec:parrot-intro}

As described in Chapter~\ref{sec:smt-motivation}, a root cause that makes 
multithreading extremely difficult to get right
is that, for decades, the contract between developers and thread runtimes has 
favored performance over correctness and grants exponentially many possible 
schedules for all inputs.  In this contract, developers use 
synchronizations to coordinate threads, while thread runtimes can
use \emph{any} of the exponentially many schedules, 
compliant with the synchronizations.  This large number of possible schedules 
make it more likely to find an efficient schedule for a workload, but ensuring
that all schedules are free of concurrency bugs is extremely challenging, and a 
single unchecked schedule may surface in the least expected moment, causing 
critical failures and vulnerabilities~\cite{therac25-investigation, 
northeast-blackout, lu:concurrency-bugs,con:hotpar12}.

Two recent techniques aim to flip this performance \vs correctness tradeoff
by reducing the number of allowed schedules. First, 
\smt~\cite{determinator:osdi10, cui:tern:osdi10, dthreads:sosp11, 
peregrine:sosp11}, which is created by my collaborators and me, aims to reduce 
the number possibles schedules on all inputs. Other researchers have also been 
building and applying \smt
systems~\cite{determinator:osdi10, dthreads:sosp11, bergan:oopsla13} to improve
software reliability. These work have shown to greatly improve the
reliability of multithreaded programs, including: (1) making reproducing
concurrency bugs much easier~\cite{cui:tern:osdi10, peregrine:sosp11}, (2)
improving the precision of program analysis~\cite{peregrine:sosp11, wu:pldi12},
leading to the detection of several new harmful data races in heavily-tested 
programs, and (3) computing a small set of schedules to cover all or most 
inputs~\cite{bergan:oopsla13}. Second, 
\dmt~\cite{dmp:asplos09,kendo:asplos09,coredet:asplos10,
dos:osdi10,grace:oopsla09} addresses the nondeterminism problem, and it focuses
on reducing the number of possible schedules on each input down to one. \dmt is
especially useful in testing and debugging multithreaded programs, however, we
have previously stated in Chapter~\ref{sec:smt-motivation} that \dmt is not as
useful as commonly perceived, and \smt is better for improving reliability of
multithreaded programs. \smt is complementary to \dmt, and several 
multithreading systems (\eg, \cite{determinator:osdi10, dthreads:sosp11, 
cui:tern:osdi10, peregrine:sosp11}) are both stable and deterministic.

However, despite these recent advances, it remains an
open challenge that whether \smt can be made simple, fast, and 
deployable on a wide range of multithreaded programs.  
This challenge is not helped much by the limited
evaluation of previous systems which often used (1) synthetic benchmarks, not
real-world programs, from incomplete benchmark suites; (2) one workload
per program; and (3) at most 8 cores (with three exceptions; see
\S\ref{sec:parrot-related}). For instance, while a previous
system \dthreads~\cite{dthreads:sosp11} achieves reasonable performance 
overhead on 14 scientific benchmark programs, we observed that
this system incurred 30$\times$ slowdown with many other programs in our
evaluation~(\S\ref{sec:parrot-eval}).

This open challenge comes from the design choices of existing \smt systems. 
Reducing schedules improves correctness
but trades performance because the schedules left may not balance each
thread's load well, causing some threads to idle unnecessarily.  Our
experiments show that ignoring load imbalance as in \dthreads
can lead to pathological
slowdown if the order of operations enforced by a schedule
\emph{serializes} the intended parallel computations
(\S\ref{sec:parrot-comparison}).  To recover performance, one method is to count
the instructions executed by each thread and select schedules that balance
the instruction counts~\cite{kendo:asplos09, coredet:asplos10,
  dmp:asplos09}, but this method is not stable because input or program
perturbations easily change the instruction counts.  The other method (we
proposed)
lets the nondeterministic OS scheduler select
a reasonably fast schedule and reuses the schedule on
compatible inputs~\cite{cui:tern:osdi10,peregrine:sosp11}, but it
requires sophisticated program analysis, complicating deployment.



To tackle this open challenge, this chapter presents \parrot, our third \smt 
(and also \dmt) system, with three contributions.
First, \parrot is a simple, practical runtime that efficiently makes
threads deterministic and stable by offering a new contract to developers.
By default, it schedules synchronizations in each thread using
round-robin, vastly reducing schedules and providing broad repeatability.
When default schedules are slow, it allows advanced developers to add
intuitive \emph{performance hints} to their code for speed.  Developers discover
where to add hints through profiling as usual, and \parrot simplifies
performance debugging by deterministically reproducing the bottlenecks.
The hints are robust to developer mistakes as they can be safely ignored
without affecting correctness. Like previous systems, \parrot's contract reduces
schedules to favor correctness over performance.  Unlike previous systems, it
allows advanced developers to optimize performance.  We believe this practical 
``meet in the middle'' contract eases writing correct, efficient programs. For 
this reason, we name this system \parrot, one of the most trainable birds.


\parrot provides two performance hint abstractions.  A \emph{soft
  barrier} encourages the scheduler to coschedule a group of threads at
given program points.  It is for performance only, and operates as a
barrier with deterministic timeouts in \parrot.  Developers use it to switch
to faster schedules without compromising determinism
when the default schedules serialize parallel
computations (\S\ref{sec:parrot-example}).  A \emph{performance critical
section}
informs the scheduler that a code region is a potential
bottleneck, encouraging the scheduler to get through the region fast.
When a thread enters a performance critical section, \parrot delegates
scheduling to the
nondeterministic OS scheduler for speed.  
Performance critical sections may trade some determinism for
performance, so they should be applied only when the schedules they add
are thoroughly checked by tools or advanced developers.
These simple abstractions
let \parrot run fast on all programs evaluated, and
may benefit other \dmt or \smt systems and classic nondeterministic
schedulers~\cite{coschedule:sigmetrics96, coschedule, partial-barrier:atc06}.


Our \parrot implementation is \pthread-compatible, simplifying deployment.
It handles many diverse constructs real-world programs depend upon such as
network operations and timeouts.  \parrot makes synchronizations outside
performance critical sections deterministic but allows nondeterministic
data races.  Although it is
straightforward to make data races deterministic in \parrot,
% with a memory commit protocol we designed, 
we deemed it not worthwhile because the cost of doing so outweighs the
benefits (\S\ref{sec:parrot-discussion}).  \parrot's determinism is similar to
\kendo's weak determinism~\cite{kendo:asplos09}, but \parrot offers stability
which Kendo lacks.


Our second contribution is an ecosystem formed by integrating \parrot with
\dbug~\cite{dbug:spin11}, an open source model checker for 
distributed and multithreaded Linux programs that systematically checks possible
schedules for bugs.
This \ecosys ecosystem is more effective than
either system alone: \dbug checks the schedules
that matter to \parrot and developers (\eg, schedules added by performance
critical sections), and \parrot greatly increases \dbug's coverage by
reducing the schedules \dbug has to check (the \emph{state space}). Our
integration is transparent to \dbug and requires only minor
changes to \parrot.  It lets \parrot effectively leverage advanced model
checking techniques~\cite{flanagan:dynamicpo, demeter:sosp11}.


Third, we quantitatively show that \parrot achieves good performance and high
model checking coverage on a diverse set of \nprog programs.  The programs
include \nrealprog real-world programs, such as \bdb~\cite{berkeleydb},
\openldap~\cite{openldap}, \redis~\cite{redis}, \mplayer~\cite{mplayer},
all \nstl parallel C++ STL algorithm 
implementations~\cite{parallel-stl} which use \openmp, and all \nimagick
parallel image processing utilities (also \openmp) in the
\imagick~\cite{imagick}
software suite.  Further, they
include \emph{all} \nbenchmarks programs in four widely used benchmark suites:
\parsec~\cite{parsec}, \phoenix~\cite{phoenix-benchmarks},
\splashx~\cite{splashx},
and \npb~\cite{npb}.  We used complete software or benchmark suites to
avoid biasing our results. The programs together cover many different
parallel programming models and idioms such as threads,
\openmp~\cite{openmp}, fork-join, map-reduce, pipeline, and workpile.  To
our knowledge, our
evaluation uses roughly \overeach more programs than any previous
\dmt or \smt evaluation, and \overcombined more than all
previous evaluations combined.
Our experiments show:
\begin{enumerate}

\item \parrot is easy to use. It averages only \hintsperprog lines of hints
  per program to get good performance, and adding hints is fast.  Of all
  \nprog programs, \nprognohints need no hints, \nproglineuphints need
  \computes which do not affect determinism, and only \nprognondethints
  programs need \nondets to trade some determinism for speed.

\item \parrot has low overhead.  At the maximum allowed (16--24) cores,
\parrot's
  geometric mean overhead is \meanrealoverhead for \nrealprog real-world
programs,
  \meanbenchoverhead for the other \nbenchmarks programs, and \meanoverhead for
all.

\item On \nprogcompared programs that two previous systems
\dthreads~\cite{dthreads:sosp11}
  and \coredet~\cite{coredet:asplos10} can both handle, \parrot's overhead is
\parrotcompoverhead whereas \dthreads's
  is \dthreadssyncoverhead and \coredet's \coredetoverhead.

\item \parrot scales well to the maximum allowed cores on our 24-core server and
  to at least three different scales/types of workloads per program.

\item \ecosys offers exponential coverage increase compared to \dbug alone.
  \parrot helps \dbug reduce the state space by \shrinkscale for
  \nprogshrink programs and increase the number of verified programs from
  \nprogverifieddbug to \nprogverifiedxxx under our test setups.
  These verified programs include all 4 real-world programs out of
  the \nprognondethints programs that need performance critical sections, so
they
  enjoy both speed and reliability.
  These quantitative reliability results help potential \parrot adopters justify
  the overhead.

\end{enumerate}

We have released \parrot's source code, entire benchmark suite, and
raw evaluation results at \github. In the remaining of this chapter,
\S\ref{sec:parrot-design} contrasts \parrot with 
previous systems on an example and gives a high-level design of
\parrot. \S\ref{sec:parrot-hints} describes the performance hint abstractions
\parrot provides, \S\ref{sec:parrot-runtime} the \parrot runtime,
and \S\ref{sec:parrot-mc} the
\parrot-\dbug ecosystem. \S\ref{sec:parrot-discussion} discusses \parrot's
determinism, \S\ref{sec:parrot-eval}
presents evaluation results, \S\ref{sec:parrot-related}
discusses related work, and \S\ref{sec:parrot-summary} concludes.
